\subsection{Gaussian forcing} \label{sec:gaussianPulse}

We consider solving the Helmholtz problem with a source term consisting of a sum of generalized Gaussians,
\ba
   f(x,y,z,t) = \sum_m a_m \, \cos( \omega_m (t-t_{0,m}) ) \, e^{ -\beta_m [ (x-x_{0,m})^2 + (y-y_{0,m})^2 + (z-z_{0,m})^2 ]^{p_m} } .
\ea 

% ---------- SQUARE --------
\subsubsection{Gaussian source in a square}
\mni
Figure~\ref{fig:gaussianSquare} shows results from CgWaveHoltz for a square. Source term: $\omega=17.17$, $a=100$, $(x_0,y_0)=(.7,.7)$, $\beta=50$, $p=1$. 
\plotTwoFigsByHeight{fig/gaussianSquareOmega17p17WaveHoltz}{fig/gaussianSquareOmega17p17}{Gaussian source, $\omega=17.17$, square64, order 4}{fig:gaussianSquare}{7cm}


\clearpage
\mni
Figure~\ref{fig:gaussianSquareOmega15} shows results from CgWaveHoltz for a square. Source term: $\omega=15$, $a=100$, $(x_0,y_0)=(.7,.7)$, $\beta=50$, $p=1$. The FPI convergence rate matches the theory.
\plotTwoFigsByHeight{fig/gaussianSquare128O4Omega15}{fig/gaussianSquare128O4Omega15FixPointMuFunction}{Gaussian source, $\omega=15$, square128, order 4. Right: $\beta$ function with distribution of eigenvalues $\lambda_j$}{fig:gaussianSquareOmega15}{7cm}

\begin{Verbatim}[fontsize=\scriptsize]
 ################ DONE KYRLOV ITERATIONS -- KSP residual=2.54e-11 (tol=1.00e-10) numberOfIterations=31 #############
CgWaveHoltz::residual: c=1, omega= 1.50000e+01, omegas= 1.49974e+01 (from symbol of D+D-), dt=4.273544e-03, adjustOmega=1
CgWaveHoltz::residual: max-res=4.010e-04 (using omega), max-res=5.004e-02 (using omega from discrete symbol)
CgWaveHoltz: omega=1.500e+01, max-res=4.010e-04, CR=0.492 CR-perPeriod=0.492, cpu= 1.58e+00(s)
CgWaveHoltz::Results saved to file gaussianWHSquare128O4Omega15Krylov.m
\end{Verbatim}


% --------------------- ANNULUS -----------------------
\clearpage
\subsubsection{Gaussian source in an annulus}
Figure~\ref{fig:gaussianAnnulusOmega9p777} shows results from CgWaveHoltz for an Annulus.
Eigenvalues for this problem were computed in \texttt{cg/ad/codes/annulusEigenvalues.maple}.
%  Source term: $\omega=17.17$, $a=100$, $(x_0,y_0)=(.7,.7)$, $\beta=50$, $p=1$. The FPI convergence rate matches the theory.
\plotTwoFigsByHeight{fig/gaussianAnnulusO4Omega9p777}{fig/gaussianAnnulusO4Omega9p777FixPointMuFunction}{Gaussian source, $\omega=9.777$, annulus8, order 4. Right: $\beta$ function with distribution of eigenvalues $\lambda_j$}{fig:gaussianAnnulusOmega9p777}{7cm}


Figure~\ref{fig:gaussianAnnulusOmega18p4} shows results from CgWaveHoltz for an Annulus for $\omega=18.4$.
It takes quite a iterations for the CR of the FPI to approach the theory; likely due to there being a few
eigenvalues near the worst case. GMRES is also quite slow. 
%  Source term: $\omega=17.17$, $a=100$, $(x_0,y_0)=(.7,.7)$, $\beta=50$, $p=1$. The FPI convergence rate matches the theory.
\plotTwoFigsByHeight{fig/gaussianAnnulusO4Omega18p4}{fig/gaussianAnnulusO4Omega18p4FixPointMuFunction}{Gaussian source, $\omega=18.4$, annulus8, order 4. Right: $\beta$ function with distribution of eigenvalues $\lambda_j$}{fig:gaussianAnnulusOmega18p4}{7cm}



% --------------------- DISK -----------------------
\clearpage
\subsubsection{Gaussian source in a disk}
Figure~\ref{fig:gaussianDiskOmega8p1} shows results from CgWaveHoltz for an disk.
Eigenvalues for this problem were computed in \texttt{cg/ad/codes/diskEigenvalues.maple}.
\plotTwoFigsByHeight{fig/gaussianDiskO4Omega8p1}{fig/gaussianDiskO4Omega8p1FixPointMuFunction}{Gaussian source, $\omega=8.1$, disk8, order 4. Right: $\beta$ function with distribution of eigenvalues $\lambda_j$}{fig:gaussianDiskOmega8p1}{7cm}

\plotOneFigByWidth{fig/gaussianDiskOmega8p1WaveHoltz}{Gaussian source, $\omega=8.1$, disk8, order 4.}{fig:gaussianDiskSolution}{6cm}{6cm}


\mni
Solution is close to the direct solver.
\begin{Verbatim}[fontsize=\scriptsize]
 ################ DONE KYRLOV ITERATIONS -- KSP residual=2.29e-11 (tol=1.00e-10) numberOfIterations=30 #############
CgWaveHoltz::residual: c=1, omega= 8.10000e+00, omegas= 8.09901e+00 (from symbol of D+D-), dt=6.687085e-03, adjustOmega=0
CgWaveHoltz::residual: max-res=1.776e-02 (using omega), max-res=9.750e-05 (using omega from discrete symbol)
CgWaveHoltz: omega=8.100e+00, max-res=1.776e-02, CR=0.486 CR-perPeriod=0.486, cpu= 3.41e+00(s)
CgWaveHoltz::Results saved to file gaussianWHDiskO4Omega8p1Krylov.m
CgWaveHoltz: max-diff=3.25e-03 (between WaveHoltz and Direct Helmholtz solution)
\end{Verbatim}


% --------------------- BOX -----------------------
\clearpage
\subsubsection{Gaussian source in a box}
Figure~\ref{fig:gaussianBoxOmega8p1} shows results from CgWaveHoltz for a 3D box.
\plotTwoFigsByHeight{fig/gaussianBoxO4Omega8p1}{fig/gaussianBoxO4Omega8p1FixPointMuFunction}{Gaussian source, $\omega=8.1$, box4, order 4. Right: $\beta$ function with distribution of eigenvalues $\lambda_j$}{fig:gaussianBoxOmega8p1}{7cm}


% --------------------- Sphere -----------------------
\clearpage
\subsubsection{Gaussian source in a sphere}
Figure~\ref{fig:gaussianSphereOmega8p1} shows results from CgWaveHoltz for a 3D solid sphere.

{\red FIX ME -- EIGENVALUES FOR A SPHERE NEEDED}

\plotTwoFigsByHeight{fig/gaussianSphereO2Omega8p1}{fig/gaussianSphereO2Omega8p1FixPointMuFunction}{Gaussian source, $\omega=8.1$, sphere4, order 2. Right: $\beta$ function with distribution of eigenvalues $\lambda_j$}{fig:gaussianSphereOmega8p1}{7cm}




% ---------------- SHAPES ---------------
\clearpage
\subsubsection{Gaussian source in a domain with three shapes}
\bni
\plotOneFigByWidth{fig/gaussianShapesOmega9p777WaveHoltz}{Gaussian pulse for shapes, $\omega=9.777$, source at $(.3,-.75)$ }{fig:gaussianPulseShapes}{6cm}{6cm}

\begin{Verbatim}[fontsize=\scriptsize]
SHAPES: omega=9.777
################ DONE KYRLOV ITERATIONS -- KSP residual=9.71e-09 (tol=1.00e-08) numberOfIterations=93 #############
CgWaveHoltz::residual: c=1, omega= 9.77700e+00, omegas= 9.77691e+00 (from symbol of D+D-), dt=1.501518e-03, adjustOmega=0
CgWaveHoltz::residual: max-res=3.500e-02 (using omega), max-res=3.382e-02 (using omega from discrete symbol)
CgWaveHoltz: omega=9.777e+00, max-res=3.500e-02, CR=0.872 CR-perPeriod=0.872, cpu= 9.54e+01(s)
CgWaveHoltz::Results saved to file cgWaveHoltz.m
CgWaveHoltz: max-diff=1.16e-02 (between WaveHoltz and Direct Helmholtz solution)
\end{Verbatim}





\clearpage
\mni
\textbf{******* OLD STUFF *******}
\begin{itemize}
  \item The Krylov method works much better than the fixed-point iteration.
\end{itemize}  

{% -------------------
%
\newcommand{\figWidth}{6cm}% height 
\newcommand{\trimfig}[2]{\trimhb{#1}{#2}{.0}{.0}{.0}{.0}}
\begin{figure}[htb]
\begin{center}
\begin{tikzpicture}[scale=1]
  \useasboundingbox (0,0.5) rectangle (14,6.);  % set the bounding box (so we have less surrounding white space)

  \draw(0.0,0.0) node[anchor=south west,xshift=-15pt,yshift=-8pt] {\trimfig{fig/gaussianForcingSquare}{\figWidth}};

  \draw(7,0.0) node[anchor=south west,xshift=-15pt,yshift=-8pt] {\trimfig{fig/gaussianForcingSquareOmega7}{\figWidth}};
% grid:
%  \draw[step=1cm,gray] (0,0) grid (14,6);
\end{tikzpicture}
\end{center}
\caption{Gaussian forcing on a square. Left forcing. Right solution for $\omega=7$. }
  \label{fig:gaussianForcingSquare}
\end{figure}
}

{% -------------------
%
\newcommand{\figWidth}{6cm}% height 
\newcommand{\trimfig}[2]{\trimhb{#1}{#2}{.0}{.0}{.0}{.0}}
\begin{figure}[htb]
\begin{center}
\begin{tikzpicture}[scale=1]
  \useasboundingbox (0,0.5) rectangle (14,6.);  % set the bounding box (so we have less surrounding white space)

  \draw(0.0,0.0) node[anchor=south west,xshift=-15pt,yshift=-8pt] {\trimfig{fig/gaussianForcingCIC}{\figWidth}};

  \draw(7,0.0) node[anchor=south west,xshift=-15pt,yshift=-8pt] {\trimfig{fig/gaussianForcingCICOmega2p56}{\figWidth}};
% grid:
%  \draw[step=1cm,gray] (0,0) grid (14,6);
\end{tikzpicture}
\end{center}
\caption{Gaussian forcing on the cic. Left forcing. Right solution for $\omega=2.56$. }
  \label{fig:gaussianForcingCIC}
\end{figure}
}


\bigskip
\textbf{Square  : Krylov iteration. SECOND-ORDER  - looks OK}
\begin{Verbatim}[fontsize=\scriptsize]
================  ORDER=2 -- looks OK 

cgwh waveHoltz.cmd -g=square512.order2.hdf -x0=.75 -y0=.75 -omega=7 -solver=krylov -tol=1e-5 -tp=1 -ad4=0 -imode=1
cgwh waveHoltz.cmd -g=square256.order2.hdf -x0=.75 -y0=.75 -omega=7 -solver=krylov -tol=1e-4 -tp=1 -ad4=0 -imode=1
cgwh waveHoltz.cmd -g=square128.order2.hdf -x0=.75 -y0=.75 -omega=7 -solver=krylov -tol=1e-3 -tp=1 -ad4=0 -imode=1

 ################ DONE KYRLOV ITERATIONS -- numberOfIterations=11 #############
CgWaveHoltz::residual: c=1, omega=7

G512
 ################ DONE KYRLOV ITERATIONS -- KSP residual=3.35e-04 (tol=1.00e-05) numberOfIterations=12 #############
CgWaveHoltz::residual: c=1, omega=7
Maximum residual = 8.888e-03

G256
 ################ DONE KYRLOV ITERATIONS -- KSP residual=3.17e-03 (tol=1.00e-04) numberOfIterations=11 #############
CgWaveHoltz::residual: c=1, omega=7
Maximum residual = 3.792e-02

G128
 ################ DONE KYRLOV ITERATIONS -- KSP residual=1.64e-03 (tol=1.00e-03) numberOfIterations=11 #############
CgWaveHoltz::residual: c=1, omega=7
Maximum residual = 1.449e-01

++++++++++++++++++++++++++++++++++++++++++++++++++++++++++++++++++++++++++++++++++++++++++++++++++++++++++++++++++++++++++++++
G128 FIXED-POINT ITERATION
FIXED-POINT -omega=7 has trouble, may be close to a resonance!! TRY -omega=6.869 ...converges NOTE numPeriods
cgwh waveHoltz.cmd -g=square128.order2.hdf -x0=.75 -y0=.75 -omega=6.869 -solver=fixedPoint -tol=1e-3 -tp=1 -ad4=0 -imode=1 -maxIt=200 -numPeriods=5

it=75:  max(|v-vOld|)=9.70e-04 (tol=0.001)
#################### DONE CgWaveHoltz: CALL cgWave : number of WaveHoltz iteration =76 #########################
Maximum residual = 4.527e-02

\end{Verbatim}


\bigskip
\textbf{Square  : Krylov iteration. FOURTH-ORDER  - not bad -- RESIDUAL LARGE AT BOUNDARY probably degrades convergence}
\begin{Verbatim}[fontsize=\scriptsize]
cgwh waveHoltz.cmd -g=square512.order4.hdf -x0=.75 -y0=.75 -omega=7 -solver=krylov -tol=1e-6 -tp=1 -ad4=0 -imode=1
cgwh waveHoltz.cmd -g=square256.order4.hdf -x0=.75 -y0=.75 -omega=7 -solver=krylov -tol=1e-5 -tp=1 -ad4=0 -imode=1
cgwh waveHoltz.cmd -g=square128.order4.hdf -x0=.75 -y0=.75 -omega=7 -solver=krylov -tol=1e-3 -tp=1 -ad4=0 -imode=1
cgwh waveHoltz.cmd -g=square64.order4.hdf -x0=.75 -y0=.75 -omega=7 -solver=krylov -tol=1e-3 -tp=1 -ad4=0 -imode=1

G512
################ DONE KYRLOV ITERATIONS -- KSP residual=1.54e-05 (tol=1.00e-06) numberOfIterations=13 #############
Maximum residual = 4.732e-05
################ DONE KYRLOV ITERATIONS -- KSP residual=7.44e-07 (tol=1.00e-07) numberOfIterations=14 #############
Maximum residual = 4.734e-05


G256
 ################ DONE KYRLOV ITERATIONS -- KSP residual=1.67e-04 (tol=1.00e-05) numberOfIterations=12 #############
Maximum residual = 5.578e-04

--- tol=1e-4 not enough
################ DONE KYRLOV ITERATIONS -- KSP residual=3.14e-03 (tol=1.00e-04) numberOfIterations=11 #############
Maximum residual = 4.488e-03


G128 
################ DONE KYRLOV ITERATIONS -- KSP residual=1.57e-03 (tol=1.00e-03) numberOfIterations=11 #############
Maximum residual = 1.392e-02


G64 
################ DONE KYRLOV ITERATIONS -- KSP residual=7.93e-04 (tol=1.00e-03) numberOfIterations=11 #############
CgWaveHoltz::residual: c=1, omega=7
Maximum residual = 1.531e-01

\end{Verbatim}

\bigskip
\bigskip
\bigskip
\textbf{CIC  : Krylov iteration. SECOND-ORDER -- NOT BAD, some wiggles in RESIDUAL near interpolation points, worse for cice16.}
\begin{Verbatim}[fontsize=\scriptsize]
cgwh waveHoltz.cmd -g=cice8.order2 -beta=10 -x0=1 -y0=1 -omega=2.56 -solver=krylov -tol=1e-5 -tp=1 -ad4=1 -imode=1
cgwh waveHoltz.cmd -g=cice4.order2 -beta=10 -x0=1 -y0=1 -omega=2.56 -solver=krylov -tol=1e-3 -tp=1 -ad4=1 -imode=1


G8
################ DONE KYRLOV ITERATIONS -- KSP residual=1.31e-03 (tol=1.00e-05) numberOfIterations=14 #############
omega=2.560e+00, Maximum residual = 7.921e-04

G4
 ################ DONE KYRLOV ITERATIONS -- KSP residual=5.26e-02 (tol=1.00e-03) numberOfIterations=12 #############
omega=2.560e+00, Maximum residual = 3.118e-03

G2 
cgwh waveHoltz.cmd -g=cice2.order2 -beta=10 -x0=1 -y0=1 -omega=2.56 -solver=krylov -tol=1e-3 -tp=1 -ad4=1 -imode=1

################ DONE KYRLOV ITERATIONS -- KSP residual=3.38e-02 (tol=1.00e-03) numberOfIterations=12 #############
omega=2.560e+00, Maximum residual = 1.640e-02

\end{Verbatim}

% gaussianForcingSquareOmega7.pdf
