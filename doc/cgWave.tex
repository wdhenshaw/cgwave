% CMAME or JCP
% \documentclass{article}
\documentclass[preprint,11pt]{elsarticle}
% \documentclass[3p,twocolumn]{elsarticle}
\usepackage[bookmarks=true,colorlinks=true,linkcolor=blue]{hyperref}

\pdfoutput=1

%% \biboptions{sort&compress}% compress bibtex entries
%\documentclass{article}
\newcommand{\old}[1]{{\color{red}old: #1}}
\newcommand{\wdh}[1]{{\color{blue}wdh: #1}}
\newcommand{\dws}[1]{{\color{orange}dws: #1}}
\usepackage{soul}% defines \st for strike-through

% \usepackage{url}	% if your citations have any \url{} commands

% \usepackage{ifpdf}
% \ifpdf
%    \usepackage[pdftex]{graphicx}
%    \usepackage{epstopdf}
%    \pdfcompresslevel=9
%    \pdfpagewidth=8.5 true in
%    \pdfpageheight=11 true in
%    \pdfhorigin=1 true in
%    \pdfvorigin=1.25 true in
% \else
%    \usepackage{graphicx}
% \fi
\usepackage[usenames,dvipsnames,svgnames,table]{xcolor}
\usepackage{color}
\usepackage{amssymb,amsmath}
\usepackage{amsthm}
\usepackage{tikz}
\usetikzlibrary{positioning}
\usetikzlibrary{decorations.pathreplacing}
\usepackage{accents}% for under-bar

% \newtheorem{theorem}{Theorem}
% \newtheorem{algorithm}{Algorithm}


\hbadness=10000 
\sloppy \hfuzz=30pt

\usepackage{calc}

\usepackage[margin=1.in]{geometry}

\usepackage{fancyvrb}% fancy verbatim

% for fancy coloured boxes: 
\usepackage{empheq}
\usepackage[most]{tcolorbox}

\newtcbox{\mymath}[1][]{%
    nobeforeafter, math upper, tcbox raise base,
    enhanced, colframe=blue!60!black,
    colback=blue!20, boxrule=1pt,
    #1}

%\usepackage[ruled]{algorithm2e}% http://ctan.org/pkg/algorithm2e
%\usepackage{algcompatible}
\usepackage{algorithm}
\usepackage[compatible]{algpseudocode}
%\usepackage{algorithm}% http://ctan.org/pkg/algorithms
%\usepackage{algpseudocode}% http://ctan.org/pkg/algorithmicx

\usepackage{listings}
\usepackage{color}
\usepackage{hyperref}
\usepackage{graphicx}
\usepackage{caption}
\usepackage{subcaption}
\usepackage{tabularx}
\usepackage{commath}
\usepackage{amsmath}
\usepackage{listings}
\usepackage{float}
\usepackage{multirow}


\lstset{
basicstyle=\footnotesize\ttfamily,
columns=flexible,
breaklines=true,
commentstyle=\color{red},
keywordstyle=\color{black}\bfseries
}


\definecolor{NBoxColor}{named}{LightSkyBlue}
\definecolor{DBoxColor}{named}{MediumTurquoise}

\newcommand{\NBox}[1]{\colorbox{NBoxColor!30}{$\displaystyle #1$}}
\newcommand{\DBox}[1]{\colorbox{DBoxColor!30}{$\displaystyle #1$}}

\usepackage{empheq}

\newcommand*\widefbox[1]{\fbox{\hspace{2em}#1\hspace{2em}}}

\input tex/defs
\input tex/trimFig.tex


\newcommand{\ignore}[1]{}
\newcommand{\nobibentry}[1]{{\let\nocite\ignore\bibentry{#1}}}

\begin{document}
\small
\definecolor{mygreen}{rgb}{0,0.6,0}
\lstdefinestyle{myc}{
basicstyle=\footnotesize\ttfamily,
keywordstyle=\bfseries\color{mygreen}
}
\lstset{style=myc}

%\maketitle
\begin{frontmatter}
\title{
CgWave: Composite Grid Wave Equation Solver. User Guide and Reference Manual
}

\author[rpi]{W.~D.~Henshaw\corref{cor1}\fnref{NSFgrantNew}}
\ead{henshw@rpi.edu}

\address[rpi]{Department of Mathematical Sciences, Rensselaer Polytechnic Institute, Troy, NY 12180, USA}

\cortext[cor1]{Department of Mathematical Sciences, Rensselaer Polytechnic Institute, 110 8th Street, Troy, NY 12180, USA.}

\fntext[DOEThanks]{This work was performed under DOE contracts from the ASCR Applied Math Program.}

\fntext[NSFgrantNew]{Research supported by the National Science Foundation under grant DMS-1519934.}

\fntext[PECASEThanks]{Research supported by a U.S. Presidential Early Career Award for Scientists and Engineers.}


% ----------------------------------
\begin{abstract}

Here is the user guide and reference manual for CgWave, a composite grid wave equation solver based
on Overture. CgWave is meant to be a simple example of an efficient Overture based PDE solver
that also runs in parallel. CgWave is also used by CgWaveHoltz to compute solutions to the
time-harmonic wave equation (i.e. Helmholtz equation) using Daniel Appel\"o's WaveHoltz approach.

\end{abstract}

\begin{keyword}
Wave equation; overset grids.
\end{keyword}

\end{frontmatter}


% ------------- Table of contents
%\clearpage
\tableofcontents
%

% -------------------------------------------------------------------------------------------------------------
\section{Introduction}

\noindent {\red \textbf{Note:} This is a work in progress. Note all features are implemented. yet. }

\bigskip 
Here is the user guide and reference manual for CgWave, a composite grid wave equation solver based
on Overture. CgWave solves the wave equation in second-order form using overset grids.
CgWave solves problems in two and three space dimensions. 

Here is a citation~\cite{pog2008a}.

\noindent\textbf{Things to do:}
\begin{enumerate}
  \item Add 4th order compatibility conditions.
  \item Extend CgWave to 3D.
\end{enumerate}  

% -------------------------------------------------------------------------------------------------------
\section{Governing Equations} \label{sec:governing}

CgWave solves the initial boundary-value problem
for the wave equation in second-order form for $u=u(\xv,t)$, 
\bse
\label{eq:WaveIBVP} 
\bat
& \p_t^2 u = c^2 \Delta u + f(\xv,t),  && \text{for $\xv\in\Omega$, $t>0$}, \\
& B u = g                             && \text{for $\xv\in\partial\Omega$, $t>0$}, \\
& u(x,0) = u_0(\xv), \quad \p_t u(\xv,0) = u_1(\xv),   && \text{for $\xv\in\partial\Omega$}. 
\eat
\ese
Here $Bu=g$ denotes some boundary conditions.
Boundary conditions include
\begin{itemize}
  \item Dirichlet
  \item Neumann 
  \item Mixed
  \item Even symmetry
  \item Radiation boundaries (far field) -- to-do. 
\end{itemize}   

% -------------------------------------------------------------------------------------------------------
\section{Numerical Scheme} \label{sec:numericalcheme}

The equations~\ref{eq:WaveIBVP} are advanced using a modified equation approach~\cite{max2006b}.
Second and fourth-order accurate schemes are currently available.

At fourth-order accuracy the scheme takes the form
\bas
\Dpt\Dmt U_\iv^n = c^2 \Delta_{4h} U_\iv^n + \f{\dt^2}{12}\, (c^2\Delta_{2h})^2 U_\iv^n  + f(\xv_\iv,t^n)  
                   + \f{\dt^2}{12}\Big( c^2\Delta_{2h} f(\xv_\iv,t^n)  + \p_t^2 f(\xv_\iv,t^n) \Big)
\eas
where $\Delta_{ph}$ is a p-th order accurate approximation to $\Delta$. 

% -------------------------------------------------------------------------------------------------------
\section{Numerical results} \label{sec:numericalResults}

Here are some numerical results. 


% ----------------------------------------------------------
\subsection{Plane Wave}

Here are errors in computing an exact plane wave solution
\bas
    u = \sin( 2\pi ( k_x y + k_y y + k_z z ) - \omega t )
\eas
with $\omega/k=c$ and $k=|\kv|$. 

\noindent{\textbf{Square - Plane Wave Order 2}
\input tables/planeWavesquareOrder2maxNormConvTable.tex

\bigskip 
\noindent{\textbf{Square -Plane Wave Order 4}
\input tables/planeWavesquareOrder4maxNormConvTable.tex

\bigskip 
\noindent{\textbf{CIC - circle in a channel - Plane Wave Order 2}
\input tables/planeWavecicOrder2maxNormConvTable.tex

\bigskip 
\noindent{\textbf{CIC - circle in a channel - Plane Wave Order 4}
\input tables/planeWavecicOrder4maxNormConvTable.tex


% ----------------------------------------------------------
\subsection{Time-periodic solution in a box (showing forcing)}

An eact solution to the forced wave equation in a rectangular box $[0,1]^d$ is
given by
\bas
   u_e(\xv,t) \eqdef \sin( k_x x) \, \sin( k_y y) \, \cos(\omega t), \\
\eas
where the forcing function is
\bas
  f(\xv,t) = \Big( -\omega^2 + c^2( k_x^2 + k_y^2) \Big) u_e(\xv,t)
\eas

\bigskip 
\noindent{\textbf{Square - Order 2}
\input tables/helmholtzsquareOrder4maxNormConvTable.tex

\bigskip 
\noindent{\textbf{Square - Order 2}
\input tables/helmholtzsquareOrder4maxNormConvTable.tex

\bigskip 
\noindent{\textbf{CIC - circle in a channel - Order 2}
\input tables/helmholtzcicOrder2maxNormConvTable.tex

\bigskip 
\noindent{\textbf{CIC - circle in a channel - Order 4}
\input tables/helmholtzcicOrder4maxNormConvTable.tex

% \clearpage
\bibliographystyle{elsart-num}
\bibliography{bib/journal-ISI,bib/henshaw,bib/henshawPapers}

\end{document}
% ******************************* END ********************************




















\clearpage
% -------------------------------------------------------------------------------------------------------------
\section{Oblique incidence scattering}

\noindent Runs done in \texttt{runs/mx/dielectricBodies} on cg6.

\noindent Matlab plots from \texttt{runs/mx/dielectricBodies/compareProbes.m} on the Mac.

\bigskip
We consider oblique incidence scattering from a finite-array of bodies as an approximation
to a periodic array.


Figure~\ref{fig:ellipseScat} shows scattering from 1, 2, and 3 ellipses as an approximation to
a periodic array of ellipses. The incident Gaussian pulse travels in direction $\kv=[1,1]^T$,
ie upward at a 45 degree angle.

Figure~\ref{fig:obliqueEllipse} shows results from an integral probe along a vertical line close to the
target ellipse (right side).
The results for two and three ellipses are relatively close. Is this good enough?
Maybe even one ellipse is good enough for our purposes.

% =================== START FIGURE =========
{% ------ GRIDS ------
% --------- START drawContour -----------
\newcommand{\drawContour}[7]{%
\begin{scope}[#1]
\draw(0.0,0) node[anchor=south west,xshift=-4pt,yshift=+0pt] {\trimfig{fig/#2}{\figWidth}};
  \draw(.75,.5) node[draw,fill=white,anchor=west,xshift=8pt,yshift=20pt] {\scriptsize #3};
 % colour bar:
\begin{scope}[xshift=6pt,yshift=-4pt]
  \draw (\xcb,\ycb) node[anchor=south west,xshift=0.25cm,yshift=.5cm,rotate=-90] {\trimfigcb{fig/colourBarLines}{\cbWidth}{\cbHeight}};
  \draw (.8,0) node[anchor=north,xshift=+3pt,yshift=+2pt] {\scriptsize $#6$};
  \draw (4.8,0) node[anchor=north,xshift=+0pt,yshift=+2pt] {\scriptsize $#7$};
\end{scope}
\end{scope}
}
% --------- END drawContour -----------
% -- for colour bar ----
\newcommand{\cbWidth}{.2cm}% colour bar width
\newcommand{\cbHeight}{4cm}% colour bar height
\newcommand{\xcb}{.5cm}% colour bar lower left corner
\newcommand{\ycb}{-.2cm}% colour bar lower left corner
\setlength{\ycbTop}{\ycb+\cbHeight}% colour bar top label position
\setlength{\ycbMid}{\ycb+\cbHeight*\real{.5}}% colour bar top label position
\newcommand{\trimfigcb}[3]{\includegraphics[width=#2, height=#3, clip, trim=17cm 2.35cm 1.65cm 2.35cm]{#1}}
%
% ========================= DRAW ================
%
\newcommand{\figWidth}{5.25cm}
\newcommand{\trimfig}[2]{\trimw{#1}{#2}{.12}{.12}{.12}{.12}}
% 
\begin{figure}[htb]
\begin{center}
% \resizebox{16cm}{!}{% START resize box
\begin{tikzpicture}[scale=1]
  \useasboundingbox (0.,0.2) rectangle (16,6);  % set the bounding box (so we have less surrounding white space)
% 
  \drawContour{xshift= -.5cm,yshift=0.0cm}{ellipseOneAngle45Ex}{$E_x$}{$\Ev$}{$t=0.3$}{$0$}{$0.8$};

  \draw[-,blue,very thick,xshift=4cm,yshift=2.8cm]  (0,-1) -- (0,1); 
  \draw[-,blue,very thick,xshift=1.2cm,yshift=2.8cm]  (0,-1) -- (0,1); 


  \drawContour{xshift=  5.cm,yshift=0.0cm}{ellipseTwoAngle45Ex}{$E_x$}{$\Ev$}{$t=0.3$}{$0$}{$0.8$};
  \begin{scope}[xshift=6.1cm,yshift=.7cm,scale=.75]
    \draw[-,blue,very thick,xshift=4cm,yshift=2.8cm]  (0,-1) -- (0,1); 
    \draw[-,blue,very thick,xshift=1.2cm,yshift=2.8cm]  (0,-1) -- (0,1); 
  \end{scope} 
  \drawContour{xshift=10.5cm,yshift=0.0cm}{ellipseThreeAngle45Ex}{$E_x$}{$\Ev$}{$t=0.3$}{$0$}{$.8$};
%   \drawContour{xshift= 0cm,yshift=4.5cm}{baMatInt2dEy}{$E_y$}{$\Ev$}{$t=0.3$}{$-.61$}{$.61$};
%   \drawContour{xshift= 0cm,yshift=0.0cm}{baMatInt2dEzRotated}{$E_z$}{$\Ev$}{$t=0.3$}{$-.44$}{$.44$};
%   \begin{scope}[xshift=8cm]
%     \drawContour{xshift= 0cm,yshift=9.0cm}{baMatInt2dExErrRotated}{$E_x$-err}{$\Hv$}{$t=0.3$}{$-.039$}{$.038$};
%     \drawContour{xshift= 0cm,yshift=4.5cm}{baMatInt2dEyErrRotated}{$E_y$-err}{$\Hv$}{$t=0.3$}{$-.0087$}{$.0099$};
%     \drawContour{xshift= 0cm,yshift=0.0cm}{baMatInt2dEzErr}{$E_z$-err}{$\Hv$}{$t=0.3$}{$-.00251$}{$.0032$};
%   \end{scope}
% 
% 
% grid:
% \draw[step=1cm,gray] (0,0) grid (16,5);
\end{tikzpicture}
% }% end resize box
\end{center}
\caption{ Oblique scattering from one or more ellipses. Blue lines on figures show locations of integral probes (approx).
    Super-grid used on all boundaries.
    }
\label{fig:ellipseScat}
\end{figure}
}
%========================= END FIGURE ===============


{% -------------------
%
\newcommand{\figWidth}{6cm}% height 
\newcommand{\trimfig}[2]{\trimh{#1}{#2}{.0}{.0}{.0}{.0}}
\begin{figure}[htb]
\begin{center}
\begin{tikzpicture}[scale=1]
  \useasboundingbox (0,0.5) rectangle (16,6.5);  % set the bounding box (so we have less surrounding white space)

  \draw(0.0,0.0) node[anchor=south west,xshift=-15pt,yshift=-8pt] {\trimfig{fig/obliqueScatteringEllipseEx}{\figWidth}};

  \draw(8,0.0) node[anchor=south west,xshift=-15pt,yshift=-8pt] {\trimfig{fig/obliqueScatteringEllipseEy}{\figWidth}};
%  \draw(7.0,0.0) node[anchor=south west,xshift=-15pt,yshift=-8pt] {\trimfig{fig/surfaceWaveSpacePeriodicRK42Ddiss}{\figWidth}};
%   \draw(7.0,0.0) node[anchor=south west,xshift=-15pt,yshift=-8pt] {\trimfig{fig/sphereScat}{\figWidth}};
% grid:
 %  \draw[step=1cm,gray] (0,0) grid (15,6);
\end{tikzpicture}
\end{center}
\caption{Oblique incidence scattering from an ellipse...}
  \label{fig:obliqueEllipse}
\end{figure}
}



% \clearpage
\bibliographystyle{elsart-num}
\bibliography{bib/journal-ISI,bib/adegdm,bib/jwb,bib/jba,bib/henshaw,bib/henshawPapers,bib/DMX,bib/AVKrefs,bib/myrefs}

\end{document}
% ******************************* END ********************************





% ==============================================================================================
\section{First-order system Maxwell's equations in one-dimension} \label{sec:fosMaxwell1d} 

To help understand properties of the more complex BA-Maxwell equations we first start with a simple model problem.
We consider the solution of the non-dispersive
Maxwell equations in one-dimension written
as a first-order system, ( TE-z mode with $v=E_y$, $w=H_z$) 
\bse
\label{eq:fosMaxwell1d}
\bat
& \eps \p_t  v = - \p_x w , \qquad&& x\in[0,1], \\
& \mu \p_t  w = - \p_x v    \qquad&& x\in[0,1].
\eat
\ese
This is a hyperbolic system for $v$ and $w$ with wave speeds $\lambda_\pm = \pm c$ with $c = 1/\sqrt{\eps\mu}$.
The characteristic variables are
\bas
& \chi_\pm = v + \eta w,   \qquad \text{(Constant along $dx/dt=\pm c$)}, \\
& \eta = \mu c = \sqrt{\f{\mu}{\eps}},  \qquad \text{(Electric impedance)}.
\eas
Well posed boundary conditions would be to assign the incoming characteristic variable,
\bas
  \chi_+(a,t) = g_a(t) , \qquad  \chi_-(b,t) = g_a(t) .
\eas
More generally one could assign the incoming characteristic in terms of the outgoing, e.g. 
\bas
  \chi_+(a,t) = \alpha \chi_-(a,t) + g_a(t)
\eas
so that one could, for example set $v(a,t)=g_a(t)$. 

Both $v$ and $w$ satisfy the second-order wave equation
\bas
     u_{tt} - c^2 u_{xx} = 0 .
\eas

\bigskip 
We discretize the equations with a method of lines approach:
\ba
 \eps \p_t V_i  &= - D_{x,m}  W_i , \\
 \mu \p_t W_i   &= - D_{x,m} V_i , 
\ea
where $V_i(t) \approx v(x_i,t)$, and $D_{x,m}$ is an m$^{th}$-order accurate centered approximation,
\bas
 &    D_{x,2} = \Dzx, \\ 
 &    D_{x,4} = \Dzx( I - \f{h^2}{6} \Dpx\Dmx) . 
\eas
Note that, neglecting boundaries, both $V_i(t)$ and $W_i(t)$ satisfy second-order semi-discrete
wave equations with wide stencils:
\ba
 \p_t^2 V_i  &= c^2  D_{x,m}^2  W_i , \\
 \p_t^2 W_i   &= c^2 D_{x,m}^2 V_i .
\ea
Note that a plus-minus mode $V_i(t) = (-1)^i v(t)$ is in the null-space of $\Dzx^2$ and hence of $D_{x,m}^2$.
This means the solution without dissipation will allow plus-minus waves (neglecting boundary effects).

\bigskip\noindent 
We consider various time-stepping approaches
\begin{itemize}
  \item RKn : some Runge Kutta scheme of order n, $n=1,2,3,4$.
  \item AB2 : second-order explicit Adams-Bashforth. 
  \item PC2 : second-order Adams predictor corrector.
  \item LF : Leap-frog.
  \item T : Taylor time-stepping of order $2$ or $4$.
\end{itemize}

\noindent
\textbf{Notes on Taylor time-stepping:}
The Taylor time-stepping is based on Taylor series in time,
\bas
    u(t+\dt) = u(t) + \dt \p_t u + \f{\dt^2}{2!} \p_t^2 u   + \f{\dt^3}{3!} \p_t^3 u + \ldots
\eas
For a general system
\bas
   \p_t \uv = A \uv 
\eas
this could be implemented as
\bas
    \uv(t+\dt) = \uv(t) + \dt A \uv + \f{\dt^2}{2!} A^2 \uv   + \f{\dt^3}{3!} A^3 \uv + \ldots
\eas
by taking powers of the operator $A$. At the spatially discrete level it is not necessary to
take powers of A. 
Consider time-derivatives of $v$ in~\eqref{eq:fosMaxwell1d}, 
\bas
 &     \p_t v = - \f{1}{\eps} \p_x w, \quad 
    \p_t^2 v = c^2 \p_x^2 v , \quad 
    \p_t^3 v = - \f{c^2}{\eps} \p_x^3 w .
\eas
We have a choice in approximating the higher spatial derivatives to either use a wide stencil (taking
a power of the discrete operator) or
a compact stencil, e.g. at second-order accuracy we could choose,  
\bas
& \p_x^2 \approx \Dzx^2, \\
\text{or} \quad & \p_x^2  \approx \Dpx\Dmx 
\eas
Currently we choose the compact approximation.
\textbf{Note} that the compact approximation will not have the plus-minus wave in the null-space.

\medskip
\noindent\textbf{Artificial dissipation.}
We add some artificial dissipation to suppress wiggles. Option I for dissipation
is a high-order filter
that is added at the end of each step (i.e. once per-time step),
\ba
   U_i^{n} \leftarrow  U_i^{n} - \f{\sigma_1}{2^{2p}} (- h^2\Dpx\Dmx)^p U_i^{n}, 
\ea
where a value of $\sigma_1=1$ will cause the plus-minus mode $U_i=(-1)^i$ to be eliminated. 
For second-order accuracy we take $p=2$ which gives a fourth-order dissipation that is globally $\Oc(h^3)$.
For fourth-order accuracy we take $p=3$ which gives a sixth-order dissipation that is globally $\Oc(h^5)$.



% \clearpage
%  \bibliographystyle{elsart-num}
% \bibliography{journal-ISI,champ,ins,waveFourierAnalysis,jwb,henshaw,henshawPapers,fsi,imex}

\subsection{Numerical results}

The Matlab code \texttt{fos.m} can be used to test the discrete approximation.

\noindent Manufactured solutions:
\begin{itemize}
 %  \item Trigonometric 
  \item Polynomial 
\end{itemize}

\noindent Exact solutions:
\begin{itemize}
\item Traveling sine wave:
    \bas
      & v_e = \sin( k_x (x - c_0 t) ) , \\
      & w_e = \f{1}{\mu c_0} v_e(x,y) . 
    \eas
  \item Pulse: 
    \bas
      & v_e = e^{-\beta (x - x_0 - c_0 t)^2} , \\
      & w_e = \f{1}{\mu c_0} v_e(x,y) . 
    \eas
\end{itemize}
where $c_0=\pm c$. 

\bigskip\noindent\textbf{Taylor, trig, order=4:}
\begin{Verbatim}[fontsize=\scriptsize]
>> fos -tf=.5 -ts=T4 -orderInSpace=4 -sol=trig -numResolutions=6 -movie=1 -bc1=d -bc2=d -kx=2 -sigma1=1
fos: ts=T4 ms=trig, eps=1, mu=1, c=1, orderInSpace=4, cfl=0.9 kx=2, kt=2, bc1=d, bc2=d
   Nx=-1, numResolutions=6, degreex=2, degreet=2, plotOption=1, plotErrors=0, numGhostToPlot=0
sigma1=1, xPulse=0.25, pulseDir=1
T44 t=5.00e-01: Nx=  20 Nt=  17 dt=2.941e-02 max-Error=6.29e-04
T44 t=5.00e-01: Nx=  40 Nt=  33 dt=1.515e-02 max-Error=3.68e-05 ratio=17.11, rate=4.10
T44 t=5.00e-01: Nx=  80 Nt=  67 dt=7.463e-03 max-Error=2.34e-06 ratio=15.72, rate=3.97
T44 t=5.00e-01: Nx= 160 Nt= 133 dt=3.759e-03 max-Error=1.46e-07 ratio=16.06, rate=4.01
T44 t=5.00e-01: Nx= 320 Nt= 267 dt=1.873e-03 max-Error=9.16e-09 ratio=15.89, rate=3.99
T44 t=5.00e-01: Nx= 640 Nt= 533 dt=9.381e-04 max-Error=5.71e-10 ratio=16.03, rate=4.00
done
\end{Verbatim}


\bigskip\noindent\textbf{RK4, trig, order=4: (periodic BC's)}
\begin{Verbatim}[fontsize=\scriptsize]
  >> fos -tf=.2 -ts=RK4 -orderInSpace=4 -sol=trig -numResolutions=6 -movie=0 -bc1=p -bc2=p -kx=2 -sigma1=1
fos: ts=RK4 ms=trig, eps=1, mu=1, c=1, orderInSpace=4, cfl=0.9 kx=2, kt=2, bc1=p, bc2=p
   Nx=-1, numResolutions=6, degreex=2, degreet=2, plotOption=1, plotErrors=0, numGhostToPlot=0
varCoeff=0, sigma1=1, xPulse=0.25, pulseDir=1, plotSurface=0
  c1=-1.000000e+00, c2=1.000000e+00
  M = [ 1.000,  0.000 ],  A=[ 0.000,  1.000]  V=[-0.707,  0.707]  W^T=[-0.707,  0.707] 
      [ 0.000,  1.000 ],    [ 1.000,  0.000]    [ 0.707,  0.707]      [ 0.707,  0.707] 
RK44 t=2.00e-01: Nx=  20 Nt=   2 dt=1.000e-01 cpu=2.00e-02 max-Error=1.98e-03
RK44 t=2.00e-01: Nx=  40 Nt=   5 dt=4.000e-02 cpu=0.00e+00 max-Error=6.70e-05 ratio=29.55, rate=4.89
RK44 t=2.00e-01: Nx=  80 Nt=  10 dt=2.000e-02 cpu=0.00e+00 max-Error=4.20e-06 ratio=15.95, rate=4.00
RK44 t=2.00e-01: Nx= 160 Nt=  20 dt=1.000e-02 cpu=1.00e-02 max-Error=2.63e-07 ratio=15.99, rate=4.00
RK44 t=2.00e-01: Nx= 320 Nt=  40 dt=5.000e-03 cpu=2.00e-02 max-Error=1.64e-08 ratio=16.00, rate=4.00
RK44 t=2.00e-01: Nx= 640 Nt=  79 dt=2.532e-03 cpu=7.00e-02 max-Error=1.06e-09 ratio=15.50, rate=3.95
done
\end{Verbatim}

\bigskip\noindent\textbf{RK4, trig, order=4: Dirichlet BC's (RK disease)}
\begin{Verbatim}[fontsize=\scriptsize]
>> fos -tf=.5 -ts=RK4 -orderInSpace=4 -sol=trig -numResolutions=6 -movie=1 -bc1=d -bc2=d -kx=2 -sigma1=1
fos: ts=RK4 ms=trig, eps=1, mu=1, c=1, orderInSpace=4, cfl=0.9 kx=2, kt=2, bc1=d, bc2=d
   Nx=-1, numResolutions=6, degreex=2, degreet=2, plotOption=1, plotErrors=0, numGhostToPlot=0
sigma1=1, xPulse=0.25, pulseDir=1
RK44 t=5.00e-01: Nx=  20 Nt=   6 dt=8.333e-02 max-Error=5.32e-03
RK44 t=5.00e-01: Nx=  40 Nt=  12 dt=4.167e-02 max-Error=5.04e-04 ratio=10.55, rate=3.40
RK44 t=5.00e-01: Nx=  80 Nt=  25 dt=2.000e-02 max-Error=6.49e-05 ratio=7.77, rate=2.96
RK44 t=5.00e-01: Nx= 160 Nt=  49 dt=1.020e-02 max-Error=1.52e-05 ratio=4.26, rate=2.09
RK44 t=5.00e-01: Nx= 320 Nt=  99 dt=5.051e-03 max-Error=3.56e-06 ratio=4.27, rate=2.09
RK44 t=5.00e-01: Nx= 640 Nt= 198 dt=2.525e-03 max-Error=8.83e-07 ratio=4.03, rate=2.01
done
\end{Verbatim}

\noindent 
\begin{Verbatim}[fontsize=\scriptsize]
blah
\end{Verbatim}


% ----------------------------------------------------------------------------------
\subsection{Engquist-Majda Radiation Boundary Conditions}

In one-dimension, the second-order Engquist-Majda radiation boundary conditions (EM2) are
\ba
   \p_x\p_t u - c\, \p_x^2 u = 0  \label{eq:EM2}
\ea
for an artificial boundary at $x=0$ (change c to $-c$ for a BC at $x=1$).
Substituting an incoming (right-moving) wave of the form $u=f(x-ct)$ into
this formula implies $\p_xf'=f''=0$ which suggests that no incoming waves are allowed except
when $f(\xi)= a_0 + a_1 \xi$. Note that $u=a_0 + a_1 x + a_2 t $ is also a solution to ~\eqref{eq:EM2}

We can discretize~\eqref{eq:EM2} to second-order accuracy using the trapezoidal rule in time, 
\ba
  \Dpt\Dzx U_i^n -  c\, \Dpx\Dpx\Big( \f{U_i^{n+1} + U_i^{n}}{2} \Big) = 0 .  \label{eq:EM2}
\ea
This formula can be solved for the value at the ghost point in terms of known values,
\bas
   U_{-1}^{n+1} = \text{Stuff} .
\eas
The value on the second ghost line is assigned by re-writting~\eqref{eq:EM2} on a grid with twice the grid
spacing ($h \rightarrow 2h$) and solving for $U_{-2}^{n+1}$.
Similarly for the the 3rd ghost line. 


\bigskip\noindent
\textbf{Notes: on EM2 radiation BC's}
\begin{enumerate}
  \item EM2 BC's were tested with the pulse exact solution. 
  \item Taylor time-steppers work fine with or without dissipation.
  \item All other time-steppers have poor behaviour with no-dissipation: $\sigma_1=0$. Reflected high-frequency errors
    generated at one end are amplified into low frequency errors at the other end! Slow decay of errors after bouncing
     back and forth (See Figure~\ref{fig:fosEM2}).
  \item All other time-steppers work ok with $\sigma_1>0$ and work best with $\sigma_1=1$. 
\end{enumerate}
% 
{% -------------------
%
\newcommand{\figWidth}{8cm}% height 
\newcommand{\trimfig}[2]{\trimh{#1}{#2}{.0}{.0}{.0}{.0}}
\begin{figure}[htb]
\begin{center}
\begin{tikzpicture}[scale=1]
  \useasboundingbox (0,0.5) rectangle (16,9);  % set the bounding box (so we have less surrounding white space)

   \draw(0.0,0.0) node[anchor=south west,xshift=-15pt,yshift=-8pt] {\trimfig{fig/fosEM2}{\figWidth}};
   \draw(8.5,0.0) node[anchor=south west,xshift=-15pt,yshift=-8pt] {\trimfig{fig/fosEM2RK44sigma1}{\figWidth}};

%%    \draw(8.5,0.0) node[anchor=south west,xshift=-15pt,yshift=-8pt] {\trimfig{fig/interfaceGDM3d}{\figWidth}};
% grid:
% \draw[step=1cm,gray] (0,0) grid (9,9);
\end{tikzpicture}
\end{center}
\caption{Propagating pulse with EM2 BC's on both sides. Left: RK44 with no dissipation.
  Right: RK44 with dissipation.
  Lines coloured by magnitude.}
  \label{fig:fosEM2}
\end{figure}
}

Figure~\ref{fig:fosEM2} (left) shows results for RK44 (RK : 4th order in time, 4th order in space)
with no dissipation, using the command 
\begin{Verbatim}[fontsize=\scriptsize]
fos -tf=5 -ts=RK4 -orderInSpace=4 -sol=pulse -Nx=100 -movie=1 -bc1=r -bc2=r -kx=2 -sigma1=0 -pulseDir=1 -xPulse=.5
\end{Verbatim}
and showing the initial (pink) pulse of magnitude 1 moving to the right. 
As it exits on the right it generating a small (dark blue) left moving high-frequency wave which 
propagates left, hits the left boundary and is converted in a larger (green) right-moving low frequency pulse ...


Figure~\ref{fig:fosEM2} (right) shows results for RK44 with dissipation. The pulse exits at the right boundary and
the solution quickly decays to size $10^{-10}$ by time $t=1$ with no reflected waves detected.

\noindent \textbf{Question:} Can we explain these behaviours? 


% ----------------------------------------------------------------------------------
\subsection{Engquist-Majda Radiation Boundary Conditions Version 2}

Following JWB's suggestion we can discretize the EM2 radiation BC~\eqref{eq:EM2}
\bas
   \p_x\p_t u - c\, \p_x^2 u = 0  , 
\eas
by introducing a new variable equal to the x-derivative of $u$  on the boundary $\phi(t) = u_x(0,t)$,
and solve
\bas
    \p_t \phi(t) = c \Dpx\Dmx U_0(t) 
\eas
as part of our MOL time-stepping scheme. At second-order accuracy, we then define
ghost points from
\bas
&    \Dpx U_i^n = \phi^n , \\
 \implies \quad &  U_{-1}^n = U_1^n - 2 \dx \phi^n . 
\eas
A second ghost line could be obtained using a $2\dx$ centered approximation
\bas
\implies \quad &  U_{-2}^n = U_2^n - 2 (2\dx) \phi^n . 
\eas
For fourth-order accuracy we could also evolve $\psi(t) = u_{xxx}(0,t)$ and
use
\bas
&    \Dzx(1-\f{\dx^2}{6}\Dpx\Dmx)  U_i^n = \phi^n , \\
&    \Dzx\Dpx\Dmx  U_i^n = \psi^n , \\
\implies\quad& \Dzx  U_i^n = \phi^n + \f{\dx^2}{6} \psi^n
\eas
to get the ghost point values.

\noindent
\textbf{Remark:} For our FOS, using $\eps v_t = - w_x$, the condition
\bas
&   \p_x\p_t v - c\, \p_x^2 v = 0  ,  
\eas
becomes, 
\bas
& - \f{1}{\eps} \p_x^2 w_{xx} - c \p_x^2 v = 0 , \\
& \p_x^2( v + \eta  w ) = 0 
\eas
where $\eta=1/(c \eps) = \mu c$ is the impedance and $\chi_+=v + \eta w$ is the incoming characteristic variable. 
Similarly , using $\mu w_t = - v_x$, 
\bas
&   \p_x\p_t w - c\, \p_x^2 w = 0  , 
\eas
becomes
\bas
& - \f{1}{\mu} \p_x^2 v_{xx} - c \p_x^2 w = 0 , \\
& \p_x^2( v + \eta w ) = 0 
\eas

A typical characteristic boundary condition would set the incoming and extrapolate the outgoing characteristic 
\bas
&    v_{-1} + \eta w_{-1} = 0 , \\
& \Dpx^2( v_{-1} - \eta w_{-1} ) =  (v_{-1} - \eta w_{-1}) -2( v_0 - \eta w_0) + (v_1 - \eta w_1) =0
\eas
giving 
\bas
v_{-1}  = ( v_0 - \eta w_0) -\half  (v_1 - \eta w_1), \\
\eta w_{-1}  = - ( v_0 - \eta w_0) + \half  (v_1 - \eta w_1), \\
\eas

\bigskip
\noindent
\textbf{Results}:  Second-order inmplementation works but requires a reduced CFL:

\bigskip\noindent\textbf{RK4, pulse, order=4: RK4 needs cfl=.25}
\begin{Verbatim}[fontsize=\scriptsize]
>> fos -tf=.7 -ts=PC2 -orderInSpace=4 -sol=pulse -numResolutions=8 -movie=0 -bc1=r -bc2=r -kx=2 -sigma1=0 -xPulse=.5 -radbcOpt=1 -cfl=.25
fos: ts=PC2 ms=trig, eps=1, mu=1, c=1, orderInSpace=4, cfl=0.25 kx=2, kt=2, bc1=r, bc2=r radbcOpt=1
   Nx=-1, numResolutions=8, degreex=2, degreet=2, plotOption=1, plotErrors=0, numGhostToPlot=0
varCoeff=0, sigma1=0, xPulse=0.5, pulseDir=1, plotSurface=0
  c1=-1.000000e+00, c2=1.000000e+00
  M = [ 1.000,  0.000 ],  A=[ 0.000,  1.000]  V=[-0.707,  0.707]  W^T=[-0.707,  0.707] 
      [ 0.000,  1.000 ],    [ 1.000,  0.000]    [ 0.707,  0.707]      [ 0.707,  0.707] 
PC24 t=7.00e-01: Nx=  20 Nt=  65 dt=1.077e-02 cpu=4.00e-02 max-Error=4.45e-01
PC24 t=7.00e-01: Nx=  40 Nt= 129 dt=5.426e-03 cpu=4.00e-02 max-Error=3.85e-02 ratio=11.54, rate=3.53
PC24 t=7.00e-01: Nx=  80 Nt= 258 dt=2.713e-03 cpu=8.00e-02 max-Error=6.05e-04 ratio=63.72, rate=5.99
PC24 t=7.00e-01: Nx= 160 Nt= 517 dt=1.354e-03 cpu=2.00e-01 max-Error=6.78e-05 ratio=8.92, rate=3.16
PC24 t=7.00e-01: Nx= 320 Nt=1034 dt=6.770e-04 cpu=9.60e-01 max-Error=8.23e-06 ratio=8.24, rate=3.04
PC24 t=7.00e-01: Nx= 640 Nt=2068 dt=3.385e-04 cpu=1.33e+00 max-Error=1.02e-06 ratio=8.05, rate=3.01
PC24 t=7.00e-01: Nx=1280 Nt=4135 dt=1.693e-04 cpu=3.75e+00 max-Error=2.41e-07 ratio=4.24, rate=2.08
PC24 t=7.00e-01: Nx=2560 Nt=8271 dt=8.463e-05 cpu=1.14e+01 max-Error=5.95e-08 ratio=4.05, rate=2.02
done
\end{Verbatim}


\clearpage

% ==============================================================================================
\section{Simple BA-Mawell equations in one-dimension}

Consider the first-order BA-type Maxwell system in one space dimension,
\bat
&    K \qv_t + C \qv_x =0 , \\
& K = \begin{bmatrix} m_{11} & m_{12} \\
  m_{21} & m_{22} \end{bmatrix},
\qquad 
 C = \begin{bmatrix} 0 & 1 \\
   1 & 0 \end{bmatrix},
\qquad
\qv = \begin{bmatrix} v \\ w \end{bmatrix} . 
\eat
where we assume $K\in\Real^{2\times2}$. 
This reduces to~\eqref{eq:fosMaxwell1d} when $m_{11} = \eps$, $m_{22}=\mu$, $m_{12}=m_{21}=0$.
Note that $C^{-1}=C$. 
Setting $A=K^{-1} C$ gives the FOS 
\ba
   \qv_t + A \qv_x =0 .  \label{eq:bafos1d}
\ea
   
\noindent Question: What are allowable $K$ ? 
To be well posed we at least need the system
to be hyperbolic and thus we need the matrix $A=K^{-1} C$ or $A^{-1}=C^{-1} K=CK$ to have real eigenvalues.
The brute force answer:
\bas
C^{-1} K = \begin{bmatrix}
    m_{21} & m_{22} \\
    m_{11} & m_{12} \end{bmatrix},
\eas
and the eigvalues satisfy
\bas
&    (\lambda-m_{21})(\lambda- m_{12}) - m_{11} m_{22}  = 0  \\
\implies &    \lambda^2 - (m_{21}+ m_{21})\lambda  + m_{12} m_{21}- m_{11} m_{22}  = 0, \\
\implies & \lambda = \f{ m_{21}+ m_{21} \pm \sqrt{ (m_{21}+ m_{21})^2 - 4 ( m_{12} m_{21}- m_{11} m_{22} )}}{2}  , \\
\implies & \lambda = \f{ m_{21}+ m_{21} \pm \sqrt{ 4 m_{11} m_{22} + (m_{21}- m_{21})^2 }}{2}  , 
\eas
For real eigenvalues we need the discriminant to be non-negative, 
\ba
4 m_{11} m_{22} + (m_{21}- m_{21})^2 \ge 0 .  \label{eq:fosDiscriminant}
\ea
This would be the necessary and sufficient condition for the system~\eqref{eq:bafos1d} to be hyperbolic.



% Combining a time-derivative and space-derivative of~\eqref{eq:bafos1d} 
% leads to the second-order system
% \bas
%   \qv_{tt} - A^2 \qv_{xx} = \qv_{tt} - M^{-1} C M^{-1} C \qv_{xx} 
% \eas
% If $M=M^*$ then $C M C= M$ 
% \bas
%    \qv_{tt} - A^2 \qv_{xx} = \qv_{tt} - M^{-2} \qv_{xx} =0
% \eas
% which has twice as many eigenvalues as ~\eqref{eq:bafos1d}. This shows the eigenvalues of $A$ are the square root of the
% eigenvalues of $M^{-2}$ or $M^2$.
% 
% \noindent \textbf{Result:} System~\eqref{eq:bafos1d} is hyperbolic if and only if $M^2$ has real non-negative eigenvalues. 


\bigskip\noindent
If $K=K^*$ is positive definite then we have an energy estimate:
\bas
\f{d}{dt} (\qv,K\qv) & = (\qv_t,K\qv) + (\qv,K\qv_t)  \\ 
    &= (K\qv_t,\qv) + (\qv,K\qv_t) \\
    &= (A\qv_x,\qv)+ (\qv,A\qv_x) \\
    & \stackrel{\rm IBP}{=} -(\qv,A\qv_x)+ (\qv,K\qv_x) = 0 
\eas
so that
\bas
   \f{d}{dt} \| K^{1/2} \qv \| = 0 , \implies \Ec(t) = \half \| K^{1/2} \qv(t) \|^2 = \text{constant}. 
\eas

\noindent Question: If $K$ is SPD, why does $C K$ have real eigenvalues?

\noindent
Answer 1: If $K$ is SPD then $m_{11}>0$ and $m_{22}>0$ and from~\eqref{eq:fosDiscriminant} we see the eigenvalues are real. 

\noindent
Answer 2: (brute force) Matrix $C$ is symmetric with eigenvalues $\pm1$. Let 
\bas
    C = V^* E V, \qquad E = \begin{bmatrix} 1 & 0 \\
                                            0 & -1 \end{bmatrix},
\eas
Let $\xv$ be an eigenvector of $C^{-1}K= C K$, 
\bas
 &   CK \xv =    V E V^* K x = \lambda \xv, \\
\implies   & E  V^* K V (V^* \xv) = \lambda V^* \xv, \\
\implies   &  V^* K V \yv  = \lambda E \yv, 
\eas
where $\yv=V^*\xv$. 
If $K$ is SPD then so is $P=V^* K V$, whence
\bas
   \yv^* P \yv = \lambda \yv^*E\yv = \lambda ( |y_1|^2 - |y_2|^2 )
\eas
The left-hand-side is positive which implies $\lambda\in\Real$.     

\medskip\noindent
Note: As long as $\yv^* P \yv \in \Real$ then the eigenvalues are real and we have a hyperbolic system.
Thuis if $K$ is symmetric negative definite then the system is  hyperbolic.


% \medskip\noindent
% Note: If $K=-K^*$ is skew-symmetric then $\lambda \in \Im$ and the problem is not well posed.

\medskip\noindent
\textbf{Question:} suppose K has a non-real complex eigenvalue -- is the problem always ill-posed ? 
Answer: No: 
the eigenvalues $\mu$ of $K$ satisfy,
\bas
   &    (\mu-m_{11})(\mu- m_{22}) - m_{12} m_{21}  = 0  \\
% \implies &    \mu^2 - (m_{11}+ m_{22})\mu  + m_{11} m_{22}- m_{12} m_{21}  = 0, \\
% \implies & \mu = \f{ m_{11}+ m_{22} \pm \sqrt{ (m_{11}+ m_{22})^2 - 4 ( m_{11} m_{22}- m_{12} m_{21} )}}{2}  , \\
\implies & \mu = \f{ m_{11}+ m_{22} \pm \sqrt{ 4 m_{12} m_{21} + (m_{22}- m_{11})^2 }}{2}  , 
\eas
The eigenvalues have an imaginary part when
\bas
         &     4 m_{12} m_{21} + (m_{22}- m_{11})^2  < 0 , \\
\implies &     m_{12} m_{21} < - \f{(m_{22}- m_{11})^2 }{4} 
\eas
It is certainly possible that the eigenvalues of $A$ are real (and the problem is well posed)
but the eigenvalues of $K$ are complex,
take e.g. $m_{11}=m_{22} >0 $ and $m_{12} = -m_{21} \ne 0$.


\bigskip\noindent 
Return now to the FOS.
\ba
   \p_t \qv + \p_x A \qv =0 , 
\ea
which we now assume to be a strongly
hyperbolic system so that $A$ has real eigenvalues and and a complete set of eigenvectors. 
Let $\lambda_k\in\Real$ and $\vv_k$  denote the eigenvalues and eigenvectors of $A$,
\bas
   A \vv_k = \lambda_k \vv_k  .
\eas
The general solution to the initial value problem is
\bas
  \qv(x,t) = \sum_k f_k(x-\lambda_k t) \vv_k 
\eas
for arbitrary functions $f_k(\xi)$.

\medskip\noindent 
\textbf{Note:} The BA-Maxwell system~\eqref{eq:bafos1d} may two positive or two negative eigenvalues
so that both fundamental waves travel in the same direction.

\medskip\noindent 
The characteristic variables are
\bas
& \chi_k(x,t) = \lv_k^T\qv(x,t) ,   \qquad \text{(constant along $dx/dt=\lambda_k$)},
\eas
where $\lv_k$ are the left eigenvectors of $A$, 
\bas
     \lv_k^T A = \lambda_k \lv_k^T 
\eas
Also define the matrix of left eigenvectors, $L$,
and the vector of characteristic variables $\chiv=[\chi_1, \chi_2]^T$, satisfying 
\bas
&      L = \begin{bmatrix} \lv_1 &  \lv_2 \end{bmatrix} , \\ 
&      A^T L = L \Lambda , \\
&      \chiv = L^T \qv , 
\eas
     
  
% ==============================================================================================
\subsection{Discrete approximations and boundary conditions}

Following the discussion in
Section~\ref{sec:fosMaxwell1d}, we discretize the equations in space with a method of lines approach, 
\ba
    & \p_t \qv_i  + A \, D_{x,m} \qv_i  = 0 ,  \\
\text{or}\quad  & \p_t \qv_i  + K^{-1} C \, D_{x,m} \  \qv_i  = 0 . 
\ea
\textbf{Note:} In multiple space dimensions with variable coefficients the system would be of the form
\bas
    & \p_t \qv_i  + A_x(\xv) \, D_{x,m} \qv_i + A_y(\xv) \, D_{y,m} \qv_i  = 0 ,  \\
\text{or}\quad    & \p_t \qv_i  + K^{-1}(\xv) C_x \, D_{x,m} \qv_i + K^{-1}(\xv) C_y \, D_{y,m} \qv_i  = 0 ,  
\eas
and it would make sense to store $K^{-1}(\xv)$ rather than storing $A_x(\xv)$ and $A_y(\xv)$. 
Also \textbf{note} that in multiple space dimensions $A_x$ and $A_y$ are no independent, the material degrees
of freedom only come from $K\in\Real^{6\times 6}$. 

\medskip\noindent
Well posed boundary conditions would be to set the incoming characteristics. The number of
boundary conditions thus depends on the signs of $\lambda_k$. 

\bigskip
The radiation boundary conditions should be imposed
on the characteristic variables.
For example, the EM2 BC for
$ \chi_{k,i}^n \approx \chi_k(x_i,t^n)$ could be discretized as 
\ba
& \Dpt\Dzx \chi_{k,i}^n -  \lambda_k \, \Dpx\Dpx\Big( \f{\chi_{k,i}^{n+1} + \chi_{k,i}^{n}}{2} \Big) = 0 ,
\ea
to define the value on the ghost point, 
\bas
   \chi_{k,-1}^{n+1} = \text{stuff}. 
\eas
Then $q$ on the ghost point is set from 
\bas
    \qv_{-1}^{n+1} = L^{-T} \chiv_{-1}^{n+1}
\eas
where $\chiv_{-1}^{n+1}$ is the vector with components $\chi_{k,-1}^{n+1}$. 

\textbf{Is this really true?}

\textbf{Not clear how to generalize to BA case ??}


\clearpage 
% -------------------------------------------------------------------
\subsection{Exact solutions - eigenmodes of a one-dimensional box}


Consider the IBVP for a hyperbolic system  in 1D, 
\bats
& \qv_t + A \qv_x = 0     \qquad&&  x\in[a,b], \quad t>0  \\
&  B_l^T\qv(a,t) = 0, \qquad B_r^T\qv(b,t)=0 , \qquad&&  t>0 \\
& \qv(x,0) = \qv_0, \qquad&&   x\in[a,b]
\eats 
where $A\in\Real^{m\times m}$, and 
where $B_l\in\Real^{p_1\times m} $ and $B_r\in\Real^{p_1\times m}$ are matrices (generally rectangular) that define the boundary conditions
and whose dimensions $p_1$ and $p_2$ depends on the number of incoming characteristics at the respective boundaries.

Let us look for time periodic solutions of the form 
\bas
    \qv = e^{i(k x -\omega t)} \vv. 
\eas
This leads to the eigenvalue condition, 
\bas
     A \vv = \f{\omega}{k} \vv 
\eas
Assume $A$ has real eigenvalues and a complete set of eigenvectors,
\bas
   A \vv_j = \lambda_j \vv_j 
\eas
then
\bas
    \f{\omega}{k_j} = \lambda_j
\eas 
The general solution is then 
\bas
  \qv &= \sum_j c_j e^{i k_j ( x  -\lambda_j t)} \vv_j  \quad 
      = \sum_j c_j e^{i (\omega/\lambda_j) ( x  -\lambda_j t)} \vv_j ,
      = \sum_{j=1}^m c_j e^{i  \omega ( x/\lambda_j  - t)} \vv_j ,
\eas
where $\omega$ is yet to be determined.
Applying the boundary conditions gives
\bas
  \sum_j c_j e^{i  \omega ( a/\lambda_j  )} B_l \vv_j  = 0 ,\\ 
  \sum_j c_j e^{i  \omega ( b/\lambda_j  )} B_r \vv_j  = 0 
\eas
which gives the system for $c_j$ 
\bas
  B \cv = 0 , \qquad \cv = \begin{bmatrix} c_1 \\ c_2 \\ \vdots \\ c_m \end{bmatrix}.
\eas
For non-trivial solutions we require $\det(B)=0$.

\bigskip\noindent
Example: Take $m=2$, $a=0$, $b=2\pi$,  and boundary conditions
\bas
  q_1(a,t)=0, \qquad q_1(b,t)=0, 
\eas
To be well posed this implies $\lambda_1>0$ and $\lambda_2 < 0$ (or vice versa) and
$\ev_1^T\vv_1 \ne 0$ and $\ev_1^T\vv_2 \ne 0 $. 
Then
\bas
B = \begin{bmatrix} v_{11} & v_{21} \\
  e^{2\pi k_1} v_{11} & e^{2\pi k_2} v_{21}
    \end{bmatrix}.
\eas
where $\vv_j = [ v_{j1}, ~ v_{j2} ]^T$. 
Setting $\det(B)=0$ gives
\bas
&    e^{2\pi(k_2-k_1)} = 1 , 
\eas
so that
\bas
   k_2 - k_1 = n, \qquad n=0,\pm1,\pm2, \ldots , 
\eas
or
\bas
  \omega_n \Big( \f{1}{\lambda_2} - \f{1}{\lambda_1}\Big)  = n , \qquad n\in\Integers , 
\eas
which defines $\omega_n$. We also have
\bas
   \cv_n = \begin{bmatrix} -v_{21}  \\  v_{11} \end{bmatrix}.
\eas
which implies that we need at least one of $v_{11}$ or $v_{21}$ to be non-zero.

The solution takes the form  **check me* 
\bas
q_1(x,t) & = -v_{21} e^{i  ( k_1 x   - \omega t)} v_{11} + v_{11} e^{i  ( k_2 x  - \omega t)} v_{21},  \\ 
       &= v_{11}v_{21} \Big(  e^{i  \half ( k_2-k_1) x } -  e^{ -i\half  ( k_2-k_1) x }   \Big) e^{ i( \half(k_2+ k_1) x  - \omega t)} , \\
       &= 2 i v_{11}v_{21} \sin( \half ( k_2-k_1) x) ~ e^{ i( \half(k_2+ k_1) x  - \omega t)} , \\
       &= 2 i v_{11}v_{21} \sin( \f{n}{2} x) ~ e^{ i( \half(k_2+ k_1) x  - \omega t)} , 
\eas
It can be seen that $q_1(x,t)$ consists of a traveling wave with wave-number $\half(k_2+ k_1)$ embedded within an envelope $\sin( \f{n}{2} x)$
which clearly satisfies the boundary conditions.

Summary:
\bas
& \omega_n \Big( \f{1}{\lambda_2} - \f{1}{\lambda_1}\Big)  = n , \qquad n\in\Integers , \\
& k_j = \f{\omega_n}{\lambda_j}, \\
& q_1(x,t) = \alpha \sin( \f{n}{2} x) ~ e^{ i( \half(k_2+ k_1) x  - \omega t)} ,  \\
& q_2(x,t) = ??
\eas
    

\newcommand{\ExEyHz}{\begin{bmatrix}
   E_x \\ E_y  \\ H_z
\end{bmatrix}}
% ==============================================================================================
\section{BA-Mawell equations: reduction to a two-dimensional model}

The non-dispersive BA-Maxwell equations in three-dimensions are
\bas
   \Am_0  \f{\p}{\p t} \EHv = \begin{bmatrix} \grad\times\Hv \\ -\grad\times\Ev \end{bmatrix}
   = C_x \p_x \EHv + C_y \p_y \EHv + C_z \p_z \EHv, \\
   \grad\cdot\Dv=0, \qquad \grad\cdot\Bv=0 
\eas
where the constant matrices $C_k$ have entries $-1$, $0$ or $1$. 

What is the simplest 2D model ? 
\bas
\Am_0  \f{\p}{\p t} \EHv =
\begin{bmatrix}
     \p_y H_z - \p_z H_y \\
     \p_z H_x - \p_x H_z \\
     \p_x H_y - \p_y H_x \\
   - \p_y E_z + \p_z E_y \\
   - \p_z E_x + \p_x E_z \\
   - \p_x E_y + \p_y E_x 
\end{bmatrix}
\eas
Setting $\p_z$ to zero, 
\bas
\Am_0  \f{\p}{\p t}
\begin{bmatrix}
   E_x \\ E_y \\ E_z \\ H_x \\ H_y \\ H_z
\end{bmatrix}
= 
\begin{bmatrix}
     \p_y H_z            \\
              - \p_x H_z \\
     \p_x H_y - \p_y H_x \\
   - \p_y E_z            \\
              + \p_x E_z \\
   - \p_x E_y + \p_y E_x 
\end{bmatrix}, \\
 \p_x E_x + \p_y E_y =0, \qquad \p_x H_x + \p_y H_y =0,
\eas
Here is a subset involving $E_x$, $E_y$ and $H_z$, 
\bas
& M   \f{\p}{\p t}
\ExEyHz
= 
\begin{bmatrix}
     \p_y H_z            \\
              - \p_x H_z \\
    -\p_x E_y + \p_y E_x 
\end{bmatrix}
=
\begin{bmatrix}
  0 & 0 & 0 \\
  0 & 0 &-1 \\
  0 &-1 & 0 
\end{bmatrix}
\p_x \ExEyHz
+ 
\begin{bmatrix}
  0 & 0 & 1 \\
  0 & 0 & 0 \\
  1 & 0 & 0 
\end{bmatrix}
\p_y \ExEyHz
, \\
&\p_x E_x + \p_y E_y =0, 
\eas


% ==============================================================================================
\section{Model BA-Mawell equations in two-dimensions}

% \newcommand{\kHat}{\hat{k}}
\newcommand{\kHatv}{\mathbf{\kHat}}
% \newcommand{\kvHat}{\kHatv}
% 
\newcommand{\uvw}{\begin{bmatrix}
   u \\ v  \\ w
\end{bmatrix}}



Consider the following system in two space dimensions, 
% ------------------ BA FOS 2D -------------
\ba
 K   \f{\p}{\p t}
\uvw
+ 
\begin{bmatrix}
  0 & 0 & 0 \\
  0 & 0 & 1 \\
  0 & 1 & 0 
\end{bmatrix}
\p_x \uvw
+ 
\begin{bmatrix}
  0 & 0 &-1 \\
  0 & 0 & 0 \\
 -1 & 0 & 0 
\end{bmatrix}
\p_y \uvw
= 0 , 
 \label{eq:baFos2d} 
\ea
or
\bas
&   K  \f{\p}{\p t} \uvw + C_x \p_x \uvw + C_y \p_y \uvw =0 ,
\eas
or
\ba
 K   \f{\p}{\p t}
\uvw
=
\begin{bmatrix} 
  \p_y w\\
  -\p_x w \\
   \p_y u  - \p_x v 
\end{bmatrix}
\ea
with $\grad\cdot\Ev=0$
\textbf{ This is wrong -- should be $\grad\cdot\Dv=0$ FIX ME.}
implying 
\bas
& \p_x u + \p_y v = 0 .  
\eas
 \textbf{Question:} Under what conditions on $K\in\Real^{3\times3}$ is this
  problem well posed?
Looking for solutions of the form
\bas
   \qv(x,y,t) = \qv_0 \, e^{i( \kv\cdot\xv - \omega t)} 
\eas
where $\kv\in\Real^3$, leads to 
\bas
& \Big(-\omega K  + k_x C_x   + k_y C_y\Big) \qv_0 = 0 , 
\eas
and we require the x and y components of $\qv_0=[u_0,v_0,w_0]^T$ to satisfy
\bas 
 k_x u_0 + k_y v_0=0.
\eas
Subsituting $k=|\kv|$, $c=\omega/k$, and $\kvHat=\kv/k$ gives 
\bas
    B \qv_0=0 
\eas
where 
\bas
B = - c K  + \kHat_x  C_x   +  \kHat_y C_y = - c K + 
\begin{bmatrix}
  0 & 0 & -\kHat_y \\
  0 & 0 & \kHat_x \\
  -\kHat_y &\kHat_x  & 0 
\end{bmatrix}
\eas

\medskip\noindent
\textbf{Note:} This can be simplified using $k_x u_0 + k_y v_0=0$ to a 2 by 2 system.
The matrix $\kHat_x  C_x   + \kHat_y C_y$ is singular and so there is always
a zero eigenvalue. The eigenvector for $c=0$ is $\qv_0=[\kHat_x, \kHat_y,0]$. 



\medskip\noindent
For non-trivial solutions we require
\ba
  \det\Big( - c K  + \kHat_x  C_x   + \kHat_y C_y\Big) = 0 ,  \\  \label{eq:wellPosed2d} 
  \det\Big( - c I  + K^{-1}(\kHat_x  C_x   + \kHat_y C_y) \Big) = 0 . 
\ea
We require that for all $\kvHat\in\Real^3$, $|\kHatv|=1$,
the roots $c$ to~\eqref{eq:wellPosed2d} must be real. 
Thus it is necessary, for example, that
$K_{-1}C_x$ and $K^{-1} C_y$ and $K^{-1}( C_x + C_y)$  have real eigenvalues.



% ==============================================================================================
\section{BA-Mawell equations: plane wave solutions}

We can look for plane wave solutions of the form
\bas
\qv = e^{ i (\kv\cdot\xv - \omega t) } \qv_0 = e^{s t} e^{ i \kv\cdot\xv }
\eas
where in general $\omega\in\Complex$ (or $s=-i\omega \in\Complex$) and $\kv\in\Complex^3$. Physics/engineers often like
to think of $\omega\in\Real$, $\omega\ge 0$,  (Fourier transform in time). 
Substituting this form
into Maxwell's equations 
leads to the eigenvalue problem
\bas
& \left\{ \omega \Km + \begin{bmatrix} 0 & \kv\times \\
  - \kv\times & 0 \end{bmatrix} \right\} \qv_0 = \zerov  , \\
& \left\{ i s \Km(s) + \begin{bmatrix} 0 & \kv\times \\
  - \kv\times & 0 \end{bmatrix} \right\} \qv_0 = \zerov  , 
\eas
Let $k\in\Complex$, and $\kvHat$ be a unit vector satisfying 
\bas
& \kv = k \kvHat , \\
& |\kvHat|=1, 
\eas
so that $k = e^{i\theta} |\kv|$.
The eigenvalue problem becomes
\bas
% &
%\left\{ \omega M +
%\begin{bmatrix}
%      0 &  0 &  0 &  0 &  0 & k_y  \\
%      0 &  0 &  0 &  0 &  0 &-k_x  \\
%      0 &  0 &  0 &-k_y& k_x&  0  \\
%      0 &  0 &-k_y&  0 &  0 &  0  \\
%      0 &  0 & k_x&  0 &  0 &  0  \\
%     k_y&-k_x&  0 &  0 &  0 &  0
%\end{bmatrix}       \right\} \qv_0 = \zerov  \\
&
\left\{ \f{\omega}{k} \Km +
\begin{bmatrix}
      0 &  0 &  0 &  0 &  -\kHat_z & \kHat_y  \\
      0 &  0 &  0 &  \kHat_z &  0 &-\kHat_x  \\
      0 &  0 &  0 &-\kHat_y& \kHat_x&  0  \\
      0 &  \kHat_z &-\kHat_y&  0 &  0 &  0  \\
      - \kHat_z & 0  & \kHat_x&  0 &  0 &  0  \\
     \kHat_y&-\kHat_x&  0 &  0 &  0 &  0
\end{bmatrix}       \right\} \qv_0 = \zerov , \\
\eas
or
\ba
& \left\{ \lambda \Km + C(\kvHat)  \right\} \qv_0 = \zerov,   \\
& \left\{ \lambda I  + \Km^{-1} C(\kvHat)  \right\} \qv_0 = \zerov,   \label{eq:BAeigenavlue} 
\ea
where
\bas
   \lambda = \f{\omega}{k}
\eas
is the wave speed.

\medskip 
\noindent
\textbf{Note:} Given $\omega$ (or $s$) equation~\eqref{eq:BAeigenavlue} is a regular eigenvalue problem for $1/k$,
  even in the dispersive case (after dividing the equation by $\omega$).

\medskip 
For any direction $\kvHat$ there will be six eigenvalues $\lambda_j$ which can be thought
of determining $\omega_j$ given $k$ or $k_j$ given $\omega$. 
Since $\grad\cdot\Dv=0$ and $\grad\cdot\Bv=0$, it follows that there will be two zero eigenvalues $\lambda=0$,
with corresponding eigenvectors given by 
\bas
\lambda_{5,6}=0, \qquad  \qv_5 = \begin{bmatrix} \kHatv \\ \zerov \end{bmatrix} , \quad
      \qv_5 = \begin{bmatrix} \zerov \\ \kHatv  \end{bmatrix} .
\eas
This leaves four additional eigenvalues $\lambda_j$, $j=1,2,3,4$.
After removing the two zero roots, the determinant condition,
\bas
    \det\Big( \lambda \Km + C(\kvHat) \Big) = 0 ,
\eas
defines a degree four polynomial in $\lambda$ (or $\omega$ or $k$),
\bas
     a_4 \lambda^4 + a_3 \lambda^3 + a_2\lambda^2 + a_1 \lambda + a_0=0.
\eas
For a given $\omega$ and a given $\kvHat$, these four eigenvalues $\lambda_j$ will define four $k_j$,
\bas
   k_j = k_j(\Km,\omega,\kvHat). 
\eas
Alternatively, given $k$ and $\kvHat$ the solution to the eigenvalue problem will define 4 values for $\omega$,
\bas
   \omega_j = \omega_j(\Km,k,\kvHat). 
\eas

For certain classes of BA materials the determinant is a degree 2 polynomial in $k^2$
\bas
    \alpha_2 (k^2)^2 + \alpha_1 (k^2) + \alpha_0 = 0, 
\eas
so that
the solutions pair up with $k_1=-k_2$ and $k_3=-k_4$. In other cases, there can be four distinct roots.

Normally we might expect two $\lambda_j $ to have real-part bigger than zero and two to be less than zero so that there
are two plane wave solutions moving in the positive direction and two in the negative direction. 
\textbf{Question:} is this always true ? Maybe, see BA book for a partial answer.


\bigskip
\textbf{Note:} If we instead define
\bas
    \kvHat = \f{\kv}{|\kv|} 
\eas
then
\bas
   \lambda_j = \f{\omega_j}{|\kv|} 
\eas
denotes the phase speed in direction $\kv$ (or $\kvHat$). Positive (negative) values for $\Re(\lambda_j)$ imply motion
in the positive (negative) $\kv$ direction.

\bigskip
The general plane-wave solution will consist of the sum of these four solutions,
\bas
&     \qv = \sum_j c_j e^{i( \kv_j\cdot\xv - \omega t)} \qv_j     
\eas
assuming there is a complete set of eigenvectors. If there is not a complete set of eigenvectors
(this can happen, see BA book ) then we need to adjust ...

% -----------------------------------------
\subsection{Growth or decay of plane wave solutions}

Consider a model problem
\bas
    u_t + c u_x = - \alpha u
\eas
with $c\in\Real$ and $\alpha\in\Complex$. 
The IVP has no growth in time provided
\bas
    \Re(\alpha) \ge 0 ,
\eas
which can be seen by Fourier transforming in space, $u(x,t) = \uHat(t) e^{i kx }$, or from an energy estimate
\bas
 \f{d}{dt} \| u \|^2 = (u,u_t) + (u_t,u) & = - (u,c u_x) - (c u_x,u) - (u,\alpha u) - (\alpha u,u ) \\
                  &=  -( u, (\alpha^*+\alpha) u) = - 2 \Re(\alpha) \| u\|^2 
\eas

For a half-plane problem on $x \ge 0$ with boundary condition at $x=0$ (signaling problem)
\bas
     u(x,0) = e^{st} , 
\eas
we can look for solutions
\bas
u(x,t) = e^{s t} e^{i k x}
\eas
which leads to the dispersion relation
\bas
   s + i c k = -\alpha 
\eas
which implies
\bas
 -2 \Re(\alpha) &= -\alpha + -\alpha ^* =  (s+ic k) + (s+ick)^* , \\
&= s+s^* + ik - i k^* = 2\Re(s) -  2c \Im(k) , \\
\implies & \Re(\alpha) =  c \Im(k) - \Re(s)
\eas
We also have the relationship
\bas
 &    \Im(s) +  c \Re(k) = - \Im(\alpha), \\
\implies  & c = -\f{ \Im(s) +\Im(\alpha)}{\Re(k)}  
\eas
Whence,
\bas
  \Re(\alpha) =  -\f{ (\Im(s) +\Im(\alpha))}{\Re(k)} \Im(k) - \Re(s) 
\eas
For no growth we still have the condition $\Re(\alpha) \ge 0$ which corresponds to
\bas
   -\f{ (\Im(s) +\Im(\alpha))}{\Re(k)} \Im(k) \ge \Re(s)
\eas
We should be able to absorb $\Im(\alpha)$ into $c$ and get the general condition (indepenedent of $c$) that
for no growth we require 
\ba
   -\f{ \Im(s)}{\Re(k)} \Im(k) \ge \Re(s)  \label{eq:stabCondition}
\ea
% This may actually be growth relative to $\Re(s)$ ?

\medskip\noindent 
The solution for $s=s_r + i s_i$ and $k=k_r + i k_i$ is given by
\bas
   u(x,t) = e^{ s_r t - k_i x} \, e^{ i ( k_r x + s_i t )} 
\eas
Along the characteristic $ k_r x + s_i t = \phi$ we have
\bas
   u = e^{ s_r t - k_i (\phi-s_i t)/k_r} e^{i\phi} = e^{ (s_r + s_i k_i/k_r )t  } e^{i\phi- (k_i/k_r) \phi}
\eas
which for no growth implies
\bas
  \boxed{  s_r + s_i \f{k_i}{k_r} \le 0 . } 
\eas
This agrees with~\eqref{eq:stabCondition}.




% ==============================================================================================
\section{BA-Mawell equations: exact solution for scattering from a planar interface}

Consider a BA plane wave  (given from the the previous section) incident
on a planar interface at $x=0$, 
\bas
 % &   \qv^i = e^{i( \kv\cdot\xv - \omega t)} \qv_0^i, \\
 &   \qv^i = e^{st} e^{i \kv\cdot\xv} \qv_0^i
\eas 
We assume the incident wave has $\omega$ and $\kv$ that satisfy the dispersion relation for the material in
the domain $x<0$ and that $\qv_0^i$ is an
eigenvector of the system. 
Let us assume that there are two transmitted waves and two reflected waves (there are always 4 waves,
but but maybe there can be 3 transmitted and one reflected or some other combination ?) 
The transmitted wave is thus of the form
\bas
%    \qv^t = \sum_{j=1}^2 \tau_j e^{i( \kvt_j\cdot\xv - \omega t)} \qv_j^t
   \qv^t = \sum_{j=1}^2 \tau_j e^{st} e^{i  \kvt_j\cdot\xv } \qv_j^t
\eas 
while the reflected wave is
\bas
%    \qv^r = \sum_{j=1}^2 r_j e^{i( \kvr_j\cdot\xv - \omega t)} \qv_j^r
   \qv^r = \sum_{j=1}^2 r_j  e^{st} e^{i \kvr_j\cdot\xv} \qv_j^r
\eas 
The full solution takes the form
\bats
\qv(\xv,t) = 
\begin{cases} 
\displaystyle  e^{st} e^{i \kv\cdot\xv} \qv_0^i + \sum_{j=1}^2 r_j e^{st} e^{i \kvr_j\cdot\xv} \qv_j^r & \text{for $x<0$}, \\
\displaystyle    \sum_{j=1}^2 \tau_j e^{st} e^{i  \kvt_j\cdot\xv} \qv_j^t  & \text{for $x>0$} , 
  \end{cases} 
\eats 
The interface conditions $[ \nv\times\Ev]=0$ and $[\nv\times\Hv]=0$ at $x=0$ imply
\ba
e^{st} e^{i\kv\cdot\xv} \tv_m^T \qv_0^i + \sum_{j=1}^2 r_j e^{st} e^{i \kvr_j\cdot\xv} \tv_m^T\qv_j^r
   = \sum_{j=1}^2 \tau_j e^{st} e^{i \kvt_j\cdot\xv}  \tv_m^T\qv_j^t,  \qquad m=1,2, \quad x=0,  \label{eq:interfaceTangential}
\ea
where $\tv_m$, $m=1,2$ are tangent vectors, and where for $\qv=[ \Ev, \Hv]^T$, 
\bas
   \tv_m\qv \eqdef \begin{bmatrix} \tv_m^T\Ev \\ \tv_m^T \Hv \end{bmatrix}
\eas
Note that~\eqref{eq:interfaceTangential} represents four equations for the four unknowns $r_j$ and $\tau_j$. 
Note that we do NOT explicitly impose the jump conditions $[\nv\cdot\Dv]=0$ and $[\nv\cdot\Bv]=0$. 
Since~\eqref{eq:interfaceTangential} holds for all $(y,z)$,
it follows that the tangential components of the
wave vectors must match, 
\bas
   k_y = \kr_{y,j} = \kt_{y,j}, \\
   k_z = \kr_{z,j} = \kt_{z,j}
\eas
for $j=1,2$. 
To determine $\kr_{x,j}$ and $\kt_{x,j}$ we must satisfy the plane wave dispersion relations in
the respective domains.

%- \noindent\textbf{Old way:} 
%- For example, given $\kr_{y,j}=k_y$ and $\kr_{z,j}=k_z$, the values for $\kr_{x,j}$, $j=1,2$
%- can be determined from two solutions to
%- the dispersion relation 
%- \ba
%- \omega =  \Omega_j(k_{x,j}) =
%-      \lambda_i(\kvr_j)  | \kvr_j | = \lambda_i(\kvr) \sqrt{ |k_{k,j}|^2 + |k_y|^2 + |k_z|^2 } . \label{eq:omegaBA}
%- \ea
%- or we need to find a root to 
%- \bas
%-      f(k_x) = \omega - \lambda_i(k_x) ~ k(k_x) = 0 
%- \eas
%- Note that this is a nonlinear equation in general and so we need to use a root finding algorithm it seems.
%- Since we want a wave propagating away from the interface we want $\kr_{x,j}<0$ (assuming $k_x>0$) and
%- $\Re(\lambda_i(\kvr_j)) >0$ (see note above about the sign of $\lambda_i$ vis-\`a-vis propagation direction). 
%- % We can choose the root that corresponds to waves propagating away from the interface.
%- The values for $\kt_{x,j}$ can be found in a similar way.

We can compute $\kr_{x,j}$ directly by solving the following generalized
eigenvalue problem for $k_x$ given $s$, $k_y$ and $k_z$, 
\bas
 &   \Big( i s\, \Km(s) + k_x C_x + k_y C_y + k_z C_z \Big) \wv = 0 .
\eas
This can be turned into a regular eignvalue problem for $1/k_x$, 
\bas
%      & -(  i s\, \Km(s) +  k_y C_y + k_z C_z ) \wv =  k_x C_x \wv , \qquad B \eqdef -(  i s\, \Km(s) +  k_y C_y + k_z C_z ),  \\
\implies &  B  \wv =  k_x C_x \wv , \qquad B \eqdef -(  i s\, \Km(s) +  k_y C_y + k_z C_z ),  \\
\implies &  B^{-1}C_x  \wv =  \f{1}{k_x} \wv \eqdef \lambda \wv ,
\eas
assuming the matrix $B$ is invertible. After finding the four relevant values for $k_x$ we choose the two that correspond
to outgoing waves, e.g. $\Re(k_x^r) <0 $ if $\Re(k_x^i>0)$. 

\bigskip The polarization vector $\pv=[\Pv, \Mv]^T$  is determined from 
\bas
   \pv(\xv,t) = \chimHat(s)\, \qv(\xv,t) ,
\eas
that is 
\bats
\pv(\xv,t) = 
\begin{cases} 
\displaystyle  e^{st} e^{i \kv\cdot\xv} \chimHat(s)\qv_0^i + \sum_{j=1}^2 r_j e^{st} e^{i \kvr_j\cdot\xv} \chimHat(s)\qv_j^r & \text{for $x<0$}, \\
\displaystyle    \sum_{j=1}^2 \tau_j e^{st} e^{i  \kvt_j\cdot\xv} \chimHat(s)\qv_j^t  & \text{for $x>0$} , 
  \end{cases} 
\eats 
The individual polarization terms used in the ADE scheme are
\bas
V_m^{ij} & =    \f{ a_{0,m}^{ij} + a_{1,m}^{ij} s }{ b_{0,m}^{ij} + b_{1,m}^{ij} s + s^2} \, q_j(\xv,t) , \\
W_m^{ij} & = \f{\p V_m^{ij}}{\p t} = s V_m^{ij} , 
\eas
where
\bas
   \chimHat_{ij} = \sum_{m=1}^{N_p^{ij}} \f{ a_{0,m}^{ij} + a_{1,m}^{ij} s }{ b_{0,m}^{ij} + b_{1,m}^{ij} s + s^2}
\eas

\bigskip 
\noindent Results from \texttt{baMatInterface.m}. There are two reflected and two transmitted waves.
\begin{Verbatim}[fontsize=\tiny]
>> baMatInterface -medium=FaradayChiral -kx=2 -ky=1 -medium2=uniaxial -plotIncident=0 -savePlot=1
Input medium=FaradayChiral, mode=-1, plotIncident=0, plotSurf=0, savePlot=1, computeOption=0 (0=s, 1=k) 
Incident direction:  [kx,ky,kz]=[2,1,0] 

---------- Left Medium=FaradayChiral -------------- 

Ks =

   2.0000 + 0.0000i   0.0000 - 0.1000i   0.0000 + 0.0000i   0.0000 - 0.2000i  -0.2000 + 0.0000i   0.0000 + 0.0000i
   0.0000 + 0.1000i   2.0000 + 0.0000i   0.0000 + 0.0000i   0.2000 + 0.0000i   0.0000 - 0.2000i   0.0000 + 0.0000i
   0.0000 + 0.0000i   0.0000 + 0.0000i   0.5000 + 0.0000i   0.0000 + 0.0000i   0.0000 + 0.0000i   0.0000 - 0.0500i
   0.0000 + 0.2000i   0.2000 + 0.0000i   0.0000 + 0.0000i   1.0000 + 0.0000i   0.0000 - 0.1000i   0.0000 + 0.0000i
  -0.2000 + 0.0000i   0.0000 + 0.2000i   0.0000 + 0.0000i   0.0000 + 0.1000i   1.0000 + 0.0000i   0.0000 + 0.0000i
   0.0000 + 0.0000i   0.0000 + 0.0000i   0.0000 + 0.0500i   0.0000 + 0.0000i   0.0000 + 0.0000i   0.7500 + 0.0000i

>>>> Compute s from k: Possible incident waves:
 omega(1)=  3.287e+00 + -2.306e-17 I,   s(1)=-2.306e-17 + -3.287e+00 I
 omega(2)=  1.820e+00 +  4.264e-17 I,   s(2)= 4.264e-17 + -1.820e+00 I
 omega(3)= -1.820e+00 +  1.547e-17 I,   s(3)= 1.547e-17 +  1.820e+00 I
 omega(4)= -3.287e+00 + -2.653e-16 I,   s(4)=-2.653e-16 +  3.287e+00 I
Choosing mode=1: s= -2.31e-17 +  -3.29e+00 I

>>> Compute complex wave-numbers given s=-2.306e-17 + -3.287e+00 I, kv=[ 2.000e+00, 1.000e+00, 0.000e+00]
 kc(1)=( -1.0000e+00, -5.3898e-17), qI=[ 0.04, 0.02, 0.80,-0.26, 0.51,-0.00]+I [-0.05, 0.09, 0.00, 0.02, 0.01,-0.15]
 kc(2)=( -1.8066e+00, -1.9997e-17), qI=[ 0.23,-0.46, 0.00, 0.10, 0.05, 0.83]+I [-0.02,-0.01,-0.15, 0.04,-0.08, 0.00]
 kc(3)=(  1.8066e+00,  6.1011e-19), qI=[-0.23, 0.46,-0.00,-0.10,-0.05, 0.83]+I [ 0.02, 0.01,-0.15,-0.04, 0.08, 0.00]
 kc(4)=(  1.0000e+00, -7.8690e-18), qI=[-0.04,-0.02, 0.80, 0.26,-0.51, 0.00]+I [ 0.05,-0.09, 0.00,-0.02,-0.01,-0.15]
Choosing mode=4 : kc= 1.000e+00 + -7.869e-18 I

 >>> incident wave vector: kxi= 2.000e+00 + -1.574e-17 I, kyi= 1.000e+00 + -7.869e-18 I, kyi= 0.000e+00 +  0.000e+00 I
Incident wave vector: qvi=[-0.23, 0.46,-0.00,-0.10,-0.05, 0.83]+ [ 0.02, 0.01,-0.15,-0.04, 0.08, 0.00] I

 +++++ reflected waves : solve for kxr given ky and kz
 lambda(1)=1/kxr = (  0.0000e+00,  0.0000e+00) (skipped).
 lambda(2)=1/kxr = (  0.0000e+00,  0.0000e+00) (skipped).
 kxr(3)=( -2.0000e+00,  4.1742e-17), qr=[ 0.04,-0.02, 0.80, 0.26, 0.51,-0.00]+I [ 0.05, 0.09, 0.00, 0.02,-0.01,-0.15]
 kxr(4)=( -3.9139e+00,  3.2068e-16), qr=[-0.13,-0.50, 0.00, 0.10,-0.03, 0.83]+I [-0.02, 0.01,-0.15,-0.02,-0.09, 0.00]
 kxr(5)=(  2.0000e+00, -6.9043e-17), qr=[-0.04,-0.02, 0.80, 0.26,-0.51,-0.00]+I [ 0.05,-0.09, 0.00,-0.02,-0.01,-0.15]
 kxr(6)=(  3.9139e+00,  2.8783e-16), qr=[-0.13, 0.50, 0.00,-0.10,-0.03, 0.83]+I [ 0.02, 0.01,-0.15,-0.02, 0.09, 0.00]

 +++++ There were 2 reflected waves.

 ============== Right Medium = uniaxial =====================

Ks2 =

     2     0     0     0     0     0
     0     1     0     0     0     0
     0     0     1     0     0     0
     0     0     0     3     0     0
     0     0     0     0     1     0
     0     0     0     0     0     1


 +++++ transmitted waves : solve for kxt given ky and kz
 lambda(1)=1/kxt = (  0.0000e+00,  0.0000e+00) (skipped).
 lambda(2)=1/kxt = (  0.0000e+00,  0.0000e+00) (skipped).
 kxt(3)=(  3.2362e+00, -2.2614e-17), qr=[-0.00,-0.00, 0.71, 0.07,-0.70,-0.00]+I [ 0.00, 0.00, 0.00,-0.00,-0.00, 0.00]
 kxt(4)=( -3.2362e+00,  2.2614e-17), qr=[-0.00,-0.00, 0.71, 0.07, 0.70,-0.00]+I [ 0.00, 0.00, 0.00,-0.00, 0.00, 0.00]
 kxt(5)=(  3.2103e+00, -2.2387e-17), qr=[-0.11, 0.69, 0.00, 0.00,-0.00, 0.71]+I [ 0.00, 0.00, 0.00, 0.00,-0.00, 0.00]
 kxt(6)=( -3.2103e+00,  2.2387e-17), qr=[-0.11,-0.69, 0.00, 0.00, 0.00, 0.71]+I [ 0.00,-0.00, 0.00, 0.00, 0.00, 0.00]

 +++++ There were 2 transmitted waves.

 ----------------- Summary -------------------
Incident     : kxi = 2.00+-0.00I qvi=[-0.23, 0.46,-0.00,-0.10,-0.05, 0.83]+ [ 0.02, 0.01,-0.15,-0.04, 0.08, 0.00] I
reflected   1: kvrx=-2.00+ 0.00I qvr=[ 0.04,-0.02, 0.80, 0.26, 0.51,-0.00]+ [ 0.05, 0.09, 0.00, 0.02,-0.01,-0.15] I
reflected   2: kvrx=-3.91+ 0.00I qvr=[-0.13,-0.50, 0.00, 0.10,-0.03, 0.83]+ [-0.02, 0.01,-0.15,-0.02,-0.09, 0.00] I
transmitted 1: kvtx= 3.24+-0.00I qvt=[-0.00,-0.00, 0.71, 0.07,-0.70,-0.00]+ [ 0.00, 0.00, 0.00,-0.00,-0.00, 0.00] I
transmitted 2: kvtx= 3.21+-0.00I qvt=[-0.11, 0.69, 0.00, 0.00,-0.00, 0.71]+ [ 0.00, 0.00, 0.00, 0.00,-0.00, 0.00] I

 +++++ reflection and transmission coefficients +++++
 r1  =[ 0.029 + -0.001 I], r2  =[-0.270 +  0.012 I], 
 tau1=[ 0.035 + -0.153 I], tau2=[ 0.855 +  0.008 I]
Saved plot baMatInterfaceFaradayChiraluniaxial.eps
\end{Verbatim}

{% -------------------
%
\newcommand{\figWidth}{9cm}% height 
\newcommand{\trimfig}[2]{\trimh{#1}{#2}{.0}{.0}{.0}{.0}}
\begin{figure}[htb]
\begin{center}
\begin{tikzpicture}[scale=1]
  \useasboundingbox (0,0.5) rectangle (16,9);  % set the bounding box (so we have less surrounding white space)

   \draw(0.0,0.0) node[anchor=south west,xshift=-15pt,yshift=-8pt] {\trimfig{fig/baMatInterfaceFaradayChiraluniaxial}{\figWidth}};
   \draw(8.5,0.0) node[anchor=south west,xshift=-15pt,yshift=-8pt] {\trimfig{fig/baMatInterfaceFaradayChiraluniaxialScat}{\figWidth}};

%%    \draw(8.5,0.0) node[anchor=south west,xshift=-15pt,yshift=-8pt] {\trimfig{fig/interfaceGDM3d}{\figWidth}};
% grid:
% \draw[step=1cm,gray] (0,0) grid (9,9);
\end{tikzpicture}
\end{center}
\caption{Scattering from a BA material interface. Left total fields. Right: scattered fields. Matlab results from baMatInterface.m}
  \label{fig:scatMatBA}
\end{figure}
}

%- Given $\omega$, we have found $\kvr_j$ and $\kvt_j$ and we also know the eigenvectors $\qv_j^r$.
%- Equation~\eqref{eq:interfaceTangential} defines four equations for the four unknowns $\tau_j$ and $r_j$. 
%- 
%- 
%- \bigskip 
%- Question: How do we compute $\p f/\p k_x$ so that we can use Newton's method
%- for a root finder?
%- \bas
%-    f'(k_x) = - \f{\p \lambda}{\p k_x} k - \lambda \f{\p k}{\p k_x}  = - \f{\p \lambda}{\p k_x} k - \lambda \f{k_x}{k}
%- \eas
%- % $\p\lambda/\p k_x$ For fixed $\omega$, we have from~\eqref{eq:omegaBA}, $\omega = \lambda k$, 
%- % and at least for $k_x\in\Real$, **check me**
%- % \bas
%- %  & \f{\p\lambda}{\p k_x} k + \lambda \f{\p k}{\p k_x} = 0  , \\
%- % \implies~ &  \f{\p\lambda}{\p k_x} k = - \lambda \f{k_x}{k}  , \\
%- % \implies~ &  \f{\p\lambda}{\p k_x} = - \lambda \f{k_x}{k^2}  .
%- % \eas
%- 
%- Question: Given an eigenvalue and eigenvector, can we compute $\p\lambda/\p k_x$ ?
%- Taking the derivative of the eigenvalue equation w.r.t $x_k$ gives 
%- \bas
%- \Big\{  \f{\p\lambda}{\p k_x} I + \Km^{-1} \p_{k_x} C(\kvHat) \Big\} \qv_0 +
%-        \Big\{ \lambda I + M^{-1}  C(\kvHat) \Big\} \p_{k_x} \qv_0 = 0 
%- \eas 




% Equation~\eqref{eq:interfaceTangential} defines four equations for the four unknowns $\tau_j$ and $r_j$. 
% Two additional interface equations are $[\nv\cdot\Dv]=0$ and $[\nv\cdot\Bv]=0$,
% \bse
% \label{eq:interfaceNormal}
% \ba
% & \begin{bmatrix} \Dv \\ \Bv \end{bmatrix}  = M \qv(\xv,t), \\
% & [\nv\cdot\Dv]=0, \\
% & [\nv\cdot\Bv]=0 .
% % & e^{i( \kv\cdot\xv - \omega t)} \nv^T M  \Ev_0^i + \sum_{j=1}^2 r_j e^{i( \kvr_j\cdot\xv - \omega t)} \nv^T M \Ev_j^r
% %    = \sum_{j=1}^2 \tau_j e^{i( \kvt_j\cdot\xv - \omega t)}  \nv^T M \Ev_j^t \qquad x=0  \\
% % & e^{i( \kv\cdot\xv - \omega t)} \nv^T M  \Hv_0^i + \sum_{j=1}^2 r_j e^{i( \kvr_j\cdot\xv - \omega t)} \nv^T M \Hv_j^r
% %    = \sum_{j=1}^2 \tau_j e^{i( \kvt_j\cdot\xv - \omega t)}  \nv^T M \Hv_j^t \qquad x=0  
% \ea
% \ese


% we can solve for $|k_x|$ from
% \bas
%    \sqrt{ |k_x|^2 + |k_y|^2 + |k_z|^2 } = \f{\lambda_j}{\omega}
% \eas

\clearpage
\input tex/baSurfaceWave


\clearpage
\input tex/characteristicBoundaryConditions

\clearpage
\input tex/superGrid

\clearpage
\input tex/periodicBoundaryConditions



\clearpage
\input tex/numericalResults

\clearpage
\input tex/timingResults


% ========================= APPENDIX =======
\appendix

\clearpage
\input tex/materialDefinitions

% \clearpage
\bibliographystyle{elsart-num}
\bibliography{bib/journal-ISI,bib/adegdm,bib/jwb,bib/jba,bib/henshaw,bib/henshawPapers,bib/DMX,bib/AVKrefs,bib/myrefs}

\end{document}
% ******************************* END ********************************
