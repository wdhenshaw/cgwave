% CMAME or JCP
% \documentclass{article}
\documentclass[preprint,11pt]{elsarticle}
% \documentclass[3p,twocolumn]{elsarticle}
\usepackage[bookmarks=true,colorlinks=true,linkcolor=blue]{hyperref}

\pdfoutput=1

%% \biboptions{sort&compress}% compress bibtex entries
%\documentclass{article}
\newcommand{\old}[1]{{\color{red}old: #1}}
\newcommand{\wdh}[1]{{\color{blue}wdh: #1}}
\newcommand{\dws}[1]{{\color{orange}dws: #1}}
\usepackage{soul}% defines \st for strike-through

% \usepackage{url}	% if your citations have any \url{} commands

% \usepackage{ifpdf}
% \ifpdf
%    \usepackage[pdftex]{graphicx}
%    \usepackage{epstopdf}
%    \pdfcompresslevel=9
%    \pdfpagewidth=8.5 true in
%    \pdfpageheight=11 true in
%    \pdfhorigin=1 true in
%    \pdfvorigin=1.25 true in
% \else
%    \usepackage{graphicx}
% \fi
\usepackage[usenames,dvipsnames,svgnames,table]{xcolor}
\usepackage{color}
\usepackage{amssymb,amsmath}
\usepackage{amsthm}
\usepackage{tikz}
\usetikzlibrary{positioning}
\usetikzlibrary{decorations.pathreplacing}
\usepackage{accents}% for under-bar

\DeclareMathOperator*{\argmin}{arg\,min}
% \newtheorem{theorem}{Theorem}
% \newtheorem{algorithm}{Algorithm}


\hbadness=10000 
\sloppy \hfuzz=30pt

\usepackage{calc}

\usepackage[margin=1.in]{geometry}

\usepackage{fancyvrb}% fancy verbatim

% for fancy coloured boxes: 
\usepackage{empheq}
\usepackage[most]{tcolorbox}

\newtcbox{\mymath}[1][]{%
    nobeforeafter, math upper, tcbox raise base,
    enhanced, colframe=blue!60!black,
    colback=blue!20, boxrule=1pt,
    #1}

%\usepackage[ruled]{algorithm2e}% http://ctan.org/pkg/algorithm2e
%\usepackage{algcompatible}
% \usepackage{algorithm}
% \usepackage[compatible]{algpseudocode}

\usepackage{algorithm,algpseudocode}
\usepackage{algorithmicx}
\algrenewcommand\alglinenumber[1]{\footnotesize #1:} % Algorithm line number font size
\newcommand{\algFontSize}{\footnotesize}

%\usepackage{algorithm}% http://ctan.org/pkg/algorithms
%\usepackage{algpseudocode}% http://ctan.org/pkg/algorithmicx

\usepackage{listings}
\usepackage{color}
\usepackage{hyperref}
\usepackage{graphicx}
\usepackage{caption}
\usepackage{subcaption}
\usepackage{tabularx}
\usepackage{commath}
\usepackage{amsmath}
\usepackage{listings}
\usepackage{float}
\usepackage{multirow}


\lstset{
basicstyle=\footnotesize\ttfamily,
columns=flexible,
breaklines=true,
commentstyle=\color{red},
keywordstyle=\color{black}\bfseries
}


\definecolor{NBoxColor}{named}{LightSkyBlue}
\definecolor{DBoxColor}{named}{MediumTurquoise}

\newcommand{\NBox}[1]{\colorbox{NBoxColor!30}{$\displaystyle #1$}}
\newcommand{\DBox}[1]{\colorbox{DBoxColor!30}{$\displaystyle #1$}}

\usepackage{empheq}

\newcommand*\widefbox[1]{\fbox{\hspace{2em}#1\hspace{2em}}}

\input tex/defs
\input tex/trimFig.tex
\usepackage{xargs}% for optional args to \newcommandx
\input tex/plotFigureMacros.tex


\newcommand{\ignore}[1]{}
\newcommand{\nobibentry}[1]{{\let\nocite\ignore\bibentry{#1}}}

\begin{document}
\small
\definecolor{mygreen}{rgb}{0,0.6,0}
\lstdefinestyle{myc}{
basicstyle=\footnotesize\ttfamily,
keywordstyle=\bfseries\color{mygreen}
}
\lstset{style=myc}

%\maketitle
\begin{frontmatter}
\title{
CgWaveHoltz: A Composite Grid Helmholtz Solver using the WaveHoltz Algorithm
}

\author[cu]{D.A. Appel\"o}
\author[rpi]{W.~D.~Henshaw\corref{cor1}\fnref{NSFgrantNew}}
\ead{henshw@rpi.edu}

\address[cu]{University of Colorado, Boulder.}
\address[rpi]{Department of Mathematical Sciences, Rensselaer Polytechnic Institute, Troy, NY 12180, USA}

\cortext[cor1]{Department of Mathematical Sciences, Rensselaer Polytechnic Institute, 110 8th Street, Troy, NY 12180, USA.}

\fntext[DOEThanks]{This work was performed under DOE contracts from the ASCR Applied Math Program.}

\fntext[NSFgrantNew]{Research supported by the National Science Foundation under grant DMS-1519934.}

\fntext[PECASEThanks]{Research supported by a U.S. Presidential Early Career Award for Scientists and Engineers.}

\begin{abstract}

  Here are notes on CgWaveHoltz Helmholtz solver that is based on solving the Helmholtz equation
  using the wave equation solver CgWave.

\end{abstract}

\begin{keyword}
Maxwell
\end{keyword}

\end{frontmatter}


% ------------- Table of contents
%\clearpage
\tableofcontents
%

% -------------------------------------------------------------------------------------------------------------
\section{Introduction}
Here are some notes on the CgWaveHoltz solver based on WaveHoltz scheme~\cite{appelo2020waveholtz} of Appel\"o et.al.
The WaveHoltz algorithm solves the time-harmonic wave equation (Helmholtz equation)
using a time-dependent wave equation solver. CgWaveHoltz uses the CgWave wave equation solver. 

% Here is a citation~\cite{pog2008a}.

\bigskip 
\noindent\textbf{\red Things to do:}
\begin{enumerate}
    \item Compatibility boundary conditions for direct Helmholtz solver.
    \item Sixth and eight-order accurate direct Helmholtz solver.
  % \item Write a direct Helmholtz solver using Oges and a direct sparse matrix solver.
  % \item Parallel version of CgWaveHoltz.
  % \item cic grid has wiggles in the residual near interp points -- track this down. Apply a high-order
  %   filter to the solution maybe?
\end{enumerate}  


% -------------------------------------------------------------------------------------------------------------
\section{WaveHoltz algorithm} \label{sec:WaveholtzAlgorithm}

The Helmholtz solution $u(\xv,t)$ satisfies,
\bse
\label{eq:Helmholtz}
\bat
 &  - \omega^2 u = c^2 \Delta u + f(\xv)  ,  \qquad  && \text{for $\xv\in\Omega$}, \\
  & B u = g(\xv)                             && \text{for $\xv\in\partial\Omega$},
\eat
\ese
Here $Bw=g$ denotes some appropriate boundary conditions.


Let $w(\xv,t)$ be a solution to the wave equation in second-order form with
a time-harmonic forcing,
\bse
\label{eq:WaveIBVP} 
\bat
& \p_t^2 w = c^2 \Delta w + f(\xv) \, e^{i\omega t} ,  && \text{for $\xv\in\Omega$, $t>0$}, \\
& B w = g(\xv) \, e^{i\omega t}                             && \text{for $\xv\in\partial\Omega$, $t>0$}, \\
& w(x,0) = v_0(\xv), \quad \p_t w(\xv,0) = v_1(\xv), \qquad  && \text{for $\xv\in\partial\Omega$}. 
\eat
\ese
and initial conditions for $u$ and $u_t$ given by $v_0(\xv)$ and $v_1(\xv)$. 


The WaveHoltz algorithm~\cite{appelo2020waveholtz} defines an affine operator $\Pi$ that
acts on the initial conditions $\vv=[v_0, \, v_1]^T$ by solving the wave equation~\eqref{eq:WaveIBVP}
with initial conditions $\vv$ over a time period $T=2\pi/\omega$ and then forming the time integral 
\newcommand{\wtvec}{\begin{bmatrix} w(\xv,t) \\ w_t(\xv,t) \end{bmatrix}}
\ba
     \Pi \vv = \f{2}{T} \int_0^T \Big( \cos(\omega t) - \f{1}{4} \Big) \, \wtvec  \, dt .
\ea
Note that $w$ implicitly depends on $v$ and $f$, $w=w(\xv,t;v,f)$. Note that $\Pi$ is an affine operator
of the form
\ba
  \Pi v = L v + b = (I-A) v + b,
\ea
where $L$ and $A$ are a linear operators.
The solution to the Helmholtz problem~\eqref{eq:Helmholtz} is found as the solution to the 
fixed point iteration
\ba
     \vv^n = \Pi \vv^{n-1} , \label{eq:fixedPoint} 
\ea
where $\vv^0=0$ (or some other initial guess). 




% -------------------------------------------------------------------------------------------------------------
\section{Using Krylov methods to accelerate the WaveHoltz algorithm} \label{sec:KrylovAcceleration} 

The fixed point iteration~\eqref{eq:fixedPoint} can be accelerated with a Krylov
method such as GMRES.
A Krylov method solves a linear system
\ba
    A x = b .
\ea
At the most basic level Krylov methods only require a function to evaluate $b$ and a function
to compute $A$ times a vector.
For the fixed point iteration~\eqref{eq:fixedPoint}, $b$ is defined as the application of $\Pi$ to
the vector $\vv^0=\zerov$ (zero initial conditions),
\ba
&      b \eqdef \Pi\, \zerov,
\ea
while the application of $A$ to an iterate $\vv^n$ is defined as 
\ba
  A \vv^n &=  \vv^n - \Pi\vv^n + b .   \label{eq:Ax}
\ea
Formula~\eqref{eq:Ax} looks a bit funny at first glance but this is the correct definition.

We use the Krylov solvers from PETSc~\cite{petsc} to solve the fixed point iteration~\eqref{eq:fixedPoint}.
PETSc has a ``matrix-free'' option that allows one to provide function to compute
a matrix-vector product. PETSc has many Krylov solvers suh as CG, GMRES, bi-CG-stab, etc.


\clearpage
\input tex/augmentedGMRES

\clearpage
\input tex/deflation

\clearpage
% -------------------------------------------------------------------------------------------------------------
\section{Numerical Results} \label{sec:numericalResults}

Here we present some numerical results.


% -------------------------------------------------------------------------------------------------------------
\input tex/trigHelmholtz.tex


% -------------------------------------------------------------------------------------------------------------
\clearpage
\input tex/gaussianHelmholtz.tex


\clearpage
% -------------------------------------------------------------------------------------------------------------
\input tex/eigenWaveResults


% ===============================================================================================
% \clearpage
\bibliographystyle{elsart-num}
\bibliography{bib/journal-ISI,bib/henshaw,bib/henshawPapers}

\end{document}
% ******************************* END ********************************
% ******************************* END ********************************


